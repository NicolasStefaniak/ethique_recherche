% Options for packages loaded elsewhere
\PassOptionsToPackage{unicode}{hyperref}
\PassOptionsToPackage{hyphens}{url}
\documentclass[
  12pt,
]{book}
\usepackage{xcolor}
\usepackage{amsmath,amssymb}
\setcounter{secnumdepth}{5}
\usepackage{iftex}
\ifPDFTeX
  \usepackage[T1]{fontenc}
  \usepackage[utf8]{inputenc}
  \usepackage{textcomp} % provide euro and other symbols
\else % if luatex or xetex
  \usepackage{unicode-math} % this also loads fontspec
  \defaultfontfeatures{Scale=MatchLowercase}
  \defaultfontfeatures[\rmfamily]{Ligatures=TeX,Scale=1}
\fi
\usepackage{lmodern}
\ifPDFTeX\else
  % xetex/luatex font selection
    \setmonofont[]{Source Code Pro}
\fi
% Use upquote if available, for straight quotes in verbatim environments
\IfFileExists{upquote.sty}{\usepackage{upquote}}{}
\IfFileExists{microtype.sty}{% use microtype if available
  \usepackage[]{microtype}
  \UseMicrotypeSet[protrusion]{basicmath} % disable protrusion for tt fonts
}{}
\makeatletter
\@ifundefined{KOMAClassName}{% if non-KOMA class
  \IfFileExists{parskip.sty}{%
    \usepackage{parskip}
  }{% else
    \setlength{\parindent}{0pt}
    \setlength{\parskip}{6pt plus 2pt minus 1pt}}
}{% if KOMA class
  \KOMAoptions{parskip=half}}
\makeatother
\usepackage{longtable,booktabs,array}
\usepackage{calc} % for calculating minipage widths
% Correct order of tables after \paragraph or \subparagraph
\usepackage{etoolbox}
\makeatletter
\patchcmd\longtable{\par}{\if@noskipsec\mbox{}\fi\par}{}{}
\makeatother
% Allow footnotes in longtable head/foot
\IfFileExists{footnotehyper.sty}{\usepackage{footnotehyper}}{\usepackage{footnote}}
\makesavenoteenv{longtable}
\usepackage{graphicx}
\makeatletter
\newsavebox\pandoc@box
\newcommand*\pandocbounded[1]{% scales image to fit in text height/width
  \sbox\pandoc@box{#1}%
  \Gscale@div\@tempa{\textheight}{\dimexpr\ht\pandoc@box+\dp\pandoc@box\relax}%
  \Gscale@div\@tempb{\linewidth}{\wd\pandoc@box}%
  \ifdim\@tempb\p@<\@tempa\p@\let\@tempa\@tempb\fi% select the smaller of both
  \ifdim\@tempa\p@<\p@\scalebox{\@tempa}{\usebox\pandoc@box}%
  \else\usebox{\pandoc@box}%
  \fi%
}
% Set default figure placement to htbp
\def\fps@figure{htbp}
\makeatother
\setlength{\emergencystretch}{3em} % prevent overfull lines
\providecommand{\tightlist}{%
  \setlength{\itemsep}{0pt}\setlength{\parskip}{0pt}}
\usepackage[]{natbib}
\bibliographystyle{plainnat}
\usepackage{booktabs}
\usepackage{amsthm}
\makeatletter
\def\thm@space@setup{%
  \thm@preskip=8pt plus 2pt minus 4pt
  \thm@postskip=\thm@preskip
}
\makeatother
\usepackage{fontspec}
\usepackage{multirow}
\usepackage{multicol}
\usepackage{colortbl}
\usepackage{hhline}
\newlength\Oldarrayrulewidth
\newlength\Oldtabcolsep
\usepackage{longtable}
\usepackage{array}
\usepackage{hyperref}
\usepackage{float}
\usepackage{wrapfig}
\usepackage{bookmark}
\IfFileExists{xurl.sty}{\usepackage{xurl}}{} % add URL line breaks if available
\urlstyle{same}
\hypersetup{
  pdftitle={Ethique de la recherche},
  pdfauthor={Nicolas Stefaniak},
  hidelinks,
  pdfcreator={LaTeX via pandoc}}

\title{Ethique de la recherche}
\author{Nicolas Stefaniak}
\date{}

\begin{document}
\maketitle

{
\setcounter{tocdepth}{1}
\tableofcontents
}
\listoffigures
\listoftables
\chapter*{}\label{section}
\addcontentsline{toc}{chapter}{}

\global\setlength{\Oldarrayrulewidth}{\arrayrulewidth}

\global\setlength{\Oldtabcolsep}{\tabcolsep}

\setlength{\tabcolsep}{2pt}

\renewcommand*{\arraystretch}{1.5}



\providecommand{\ascline}[3]{\noalign{\global\arrayrulewidth #1}\arrayrulecolor[HTML]{#2}\cline{#3}}

\begin{longtable}[c]{|p{5.35in}}



\ascline{2pt}{000000}{1-1}

\multicolumn{1}{>{\centering}m{\dimexpr 5.35in+0\tabcolsep}}{\textcolor[HTML]{000000}{\fontsize{18}{18}\selectfont{\global\setmainfont{Arial}{\textbf{L'éthique\ de\ la\ recherche\ et\ des\ chercheurs}}}}} \\

\ascline{2pt}{000000}{1-1}



\end{longtable}



\arrayrulecolor[HTML]{000000}

\global\setlength{\arrayrulewidth}{\Oldarrayrulewidth}

\global\setlength{\tabcolsep}{\Oldtabcolsep}

\renewcommand*{\arraystretch}{1}

\textbf{Nicolas Stefaniak}

Maître de conférences en Psychologie

\chapter{Quand est-il légitime de mener une recherche ?}\label{quand-est-il-luxe9gitime-de-mener-une-recherche}

\section{Introduction}\label{introduction}

La recherche clinique renvoie aux recherches sur des êtres humains qui évaluent l'impact d'interventions avec pour objectif de déterminer si ces interventions peuvent aider à améliorer la santé et le bien-être des humains. Formulé autrement, il s'agit des recherches qui cherchent à identifier les meilleures méthodes pour traiter, soigner et prévenir les maladies. Cependant, l'histoire de James Lund amène à se questionner sur le sort des marins qui n'ont pas guéri, voire qui sont décédés. On accepte assez aisément l'idée que les médecins essaient de choisir le traitement qui leur semble le plus approprié pour leurs patients, mais ce n'est pas ce que Lind a fait \citep{sep-clinical-research}. Il avait l'intuition que l'eau de mer n'était pas une bonne idée, il en a pourtant donné à deux des marins pour savoir si c'était lui ou ceux qui lui avaient recommandé l'eau de mer qui avaient raison. La question la plus fondamentale est de savoir s'il a sacrifié ces marins sous sa responsabilité au profit des autres ?
En répartissant les marins entre différentes conditions expérimentales, Lind ne s'interrogeait pas sur le meilleur traitement qu'il considérait être pour eux, mais se demandait quelle serait la meilleure méthode pour comparer les différentes approches. Bien plus, ce n'est pas parce qu'il pensait que chaque traitement avait les mêmes chances d'être efficace qu'il les a comparés, mais parce qu'il était convaincu que plusieurs interventions pouvaient être nuisibles, et que son approche était le meilleur moyen pour le montrer. En répartissant les participants de manière aléatoire, la recherche clinique s'écarte clairement de la pratique clinique puisque les intérêts du patient sont sacrifiés au profit de la qualité des informations récoltées \citep{sep-clinical-research}.

Si on comprend aisément que la recherche clinique permet d'identifier les meilleures méthodes permettant de lutter contre une maladie, il est rare de pouvoir identifier au travers d'une seule étude une méthode efficace et sans risque permettant d'atteindre cet objectif.

Dès lors, il est nécessaire de se demander quand il est légitime ou non de réaliser un essai clinique.
Si nous nous rappelons que Lund, quand il a mené son expérience sur le scorbut, était avant tout le médecin, le clinicien, à bord du navire, il est raisonnable de considérer que la recherche clinique devrait être régie par l'éthique des soins cliniques et que les méthodes de recherche ne devraient pas diverger des méthodes acceptables en soins cliniques \citep{sep-clinical-research}. Selon cette approche, on ne devrait pas proposer à un patient de participer à un essai clinique, dont on ne sait pas si le nouveau traitement est efficace et dont une partie des patients seront possiblement traités à l'aide d'un placebo, si un traitement efficace existe ou si la recherche peut exposer le patient à des risques qui ne sont pas présents avec le traitement usuel. Ainsi, l'idée est qu'une personne présentant une maladie a droit à un traitement qui est dans son intérêt médical \citep{Rothman2000}. On pourrait même aller plus loin. Pour certains, il serait inacceptable, pour un médecin, de mener ou d'inciter ses patients à participer à un essai clinique, à moins que cet essai soit conforme aux intérêts médicaux des patients. Sans ce prérequis minimal, cela reviendrait à fournir un traitement médical de moindre qualité et à violer les obligations que les cliniciens ont \citep{Miller2006}. Ceci se justifie également par le fait que les normes de la recherche clinique découlent en grande partie des obligations qui incombent aux cliniciens.

Partir du postulat que la recherche clinique devrait satisfaire aux normes de la médecine clinique a pour avantage de protéger les participants aux recherches sur les individus et rassure le public sur le fait qu'ils sont effectivement protégés. Si les participants à la recherche sont traités conformément à leurs intérêts médicaux, on peut raisonnablement s'assurer que les améliorations en médecine clinique ne se seront pas obtenues à leur détriment.

\section{Les aspects éthiques versus la législation.}\label{les-aspects-uxe9thiques-versus-la-luxe9gislation.}

Il est important ici de distinguer les questions éthiques qui peuvent émerger lorsqu'on veut réaliser un essai contrôlé randomisé des conditions légales qui autorisent un chercheur à réaliser ce type d'études.

Cette partie rappelle les éléments règlementaires qui valent en France au moment où ces lignes sont écrites.

Deux règlementations doivent être évoquées. La première est la loi sur les recherches impliquant la personne humaine, également connue sous le nom de loi Jardé. Elle définit le cadre règlementaire dans lequel il est légitime de réaliser une recherche où des humains vont prendre part à l'étude. La seconde est la règlementation RGPD qui va concerner virtuellement toutes les études qui sont menées, puisque cette règlementation s'applique dès lors qu'un consentement est demandé.

Concernant les recherches impliquant la personne humaine, il existe un imbroglio sur le fait de savoir si les recherches en psychologie relèvent ou non de cette loi. Pour l'illustrer, on peut reprendre Dupont \citeyearpar{Dupont2019} qui souligne que, quand la ministre Vallaud-Belkacem a été interpellée sur le fait de savoir si les recherches en psychologie requièrent un avis de comité de protection des personnes, elle indique que ce n'est pas la vocation de la loi Jardé, excepté si elles répondent à la définition de l'article L1121-1 du code de la santé publique.

Pour comprendre cette réponse, il est bon de retranscrire les éléments importants de cet \href{https://www.legifrance.gouv.fr/codes/article_lc/LEGIARTI000046125746}{article}.

Ainsi, le premier paragraphe indique que ``les recherches organisées et pratiquées sur l'être humain en vue du développement des connaissances \textbf{biologiques ou médicales} sont autorisées dans les conditions prévues au présent livre''.

Dans le paragraphe suivant, il est précisé qu'il existe trois catégories de recherches impliquant la personne humaine :

\begin{itemize}
\item
  Les recherches interventionnelles qui comportent une intervention sur la personne non justifiée par sa prise en charge habituelle ;
\item
  Les recherches interventionnelles qui ne comportent que \textbf{des risques et des contraintes minimes}, dont la liste est fixée par arrêté du ministre chargé de la santé, après avis du directeur général de \textbf{l'Agence nationale de sécurité du médicament et des produits de santé} ;
\item
  Les recherches non interventionnelles qui ne comportent aucun risque ni contrainte dans lesquelles \textbf{tous les actes sont pratiqués et les produits utilisés de manière habituelle}.
\end{itemize}

Au regard de ces passages, on pourrait penser que les recherches en psychologie, y compris les psychothérapies ne relèvent pas de cette loi puisqu'on pourrait aisément considérer que les psychothérapies ne comportent que des risques et des contraintes minimes. Ainsi, la fin de la phrase mettant l'accent sur l'Agence de sécurité du médicament incite à penser que cela ne concerne donc pas la psychologie. De même, pour les recherches de catégorie 3, on identifie que l'article fait référence à des actes et des produits utilisés. Il est raisonnable de considérer que la psychothérapie n'est ni un acte ni un produit.

Cette vision est étayée par le contenu de l'article 2 qui indique que ``\textbf{Sont des recherches impliquant la personne humaine} au sens du présent titre \textbf{les recherches} organisées et pratiquées sur des personnes volontaires saines ou malades, \textbf{en vue du développement des connaissances biologiques ou médicales} qui visent à évaluer :

« 1° Les mécanismes de fonctionnement de l'organisme humain, normal ou pathologique ;

« 2° L'efficacité et la sécurité de la réalisation d'actes ou de l'utilisation ou de l'administration de produits dans un but de diagnostic, de traitement ou de prévention d'états pathologiques.

« II.-1° \textbf{Ne sont pas des recherches impliquant la personne humaine} au sens du présent titre les recherches qui, bien qu'organisées et pratiquées sur des personnes saines ou malades, n'ont pas pour finalités celles mentionnées au I, et qui visent :
\ldots{}

« d) \textbf{A réaliser des expérimentations en sciences humaines et sociales dans le domaine de la santé}.

La psychologie étant une science humaine, on pourrait considérer aisément que cet article de loi ne concerne pas la psychologie. Par ailleurs, on ne peut pas réellement considérer que les recherches en psychologie contribuent au développement des connaissances biologiques (bien que les situations où on propose par exemple de faire du sport à des personnes pour lutter contre un trouble pourrait rentrer dans cette catégorie), quant aux connaissances médicales, cela est sujet à appréciation : pour faire simple personne n'est d'accord. Ainsi, on pourraientt évoquer la bonne foi pour se prémunir de devoir se conformer à cette loi.

Néanmoins, arrêter la lecture à ces paragraphes serait réducteur et représenterait sans aucun doute une lecture biaisée de l'article de loi. En effet, \href{https://www.legifrance.gouv.fr/loda/id/JORFTEXT000036805796}{l'Annexe 1} précise les interventions qui relèvent d'une recherche mentionnée au point 2, c'est-à-dire les recherches interventionnelles qui ne comportent que des risques et des contraintes minimes. Dans cette annexe, on retrouve au moins 3 points qui relèvent clairement de la psychologie :

\begin{enumerate}
\def\labelenumi{\arabic{enumi})}
\item
  Techniques de recueil : imagerie ne comportant pas d'injection de produits de contraste ou de médicaments radiopharmaceutiques, notamment par radiographie standard, scanners, imagerie par résonance magnétique (IRM).
\item
  Techniques de psychothérapie et de thérapies cognitivo-comportementales dans le cadre d'un protocole établi et validé par un professionnel disposant des compétences appropriées dans ce domaine.
\item
  Entretiens, observations et questionnaires dont les résultats, conformément au protocole, peuvent conduire à la modification de la prise en charge médicale habituelle du participant et ne relevant pas de ce fait de la recherche mentionnée au 3° de l'article L. 1121-1 du code de la santé publique.
\end{enumerate}

La conséquence directe de relever du point 2 de l'article L. 1121-1 est que les recherches non interventionnelles ne peuvent être mises en œuvre qu'après avis favorable du comité de protection des personnes, comme stipulé à l'\href{https://www.legifrance.gouv.fr/codes/section_lc/LEGITEXT000006072665/LEGISCTA000006170998/\#LEGISCTA000032722874}{Article 4}.

Il apparaît dès lors que des études utilisant l'IRM, les études utilisant une psychothérapie et les études qui s'appuient sur des questionnaires dont les résultats peuvent modifier la prise en charge médicale relèvent du point 2. Toutes ces méthodes relèvent du type de recherche qu'on peut mener en psychologie. Pour les personnes qui n'identifieraientt pas de quelle manière l'utilisation d'un questionnaire en psychologie pourrait modifier la prise en charge, il suffit de prendre l'exemple d'un questionnaire de dépression ou d'anxiété qui amène à détecter chez une personne qu'elle souffre d'une dépression sévère ou d'un trouble anxieux sévère. Dans un cas ou dans l'autre, ce questionnaire suffirait à rentrer dans cette catégorie. Ici se pose la question de savoir ce qui conduit ou non à la prise en charge médicale. Par exemple, l'observation du visage d'une personne peut suffire à identifier toute une série de troubles : trisomie, syndrome d'alcoolisation féotale, atteinte hépatique, obésité morbide, eczema, acné sévère\ldots{} Il s'ensuit que le fait même de lancer une recherche relèverait de cette catégorie si on pousse le raisonnement à son paroxysme. C'est l'interprétation, sans doute erronée, que certaines personnes font de cette loi. Les implications de cette vision sont que quasiment toutes les recherches avec des êtres humains relèvent de cette catégorie, puisqu'il suffit de rencontrer les participants pour identifier ces éléments. Il s'ensuit que le point d de l'article II serait en contradiction avec les annexes. On pourrait toujours évoqué le fait que l'objectif de la recherche doit avoir la priorité, et si l'objectif de la recherche n'est pas d'améliorer les conditions de santé/de modifier la prise en charge des participants, alors cela ne relève pas de la RIPH. Le problème auquel on est alors confronté est qu'une étude relèverait ou non de la loi RIPH selon l'habillage qu'on en fait : Je fais passer le Mini Mental State Examination \citep{Folstein1975} pour m'assurer d'un fonctionnement cognitif minimum des personnes, cela ne relèverait pas de la loi RIPH ; je fais passer le Mini Mental State Examination \citep{Folstein1975} pour exclure les personnes susceptibles d'être atteintes d'une maladie neurodégénérative, cela relèverait de la RIPH.

Avant de continuer le raisonnement, nous permettons un petit aparté pour attirer l'attention du lecteur sur le fait que la législation, dans son texte principal et dans ses annexes, ne spécifie en aucune manière les risques qu'il est légitime ou non de faire encourir à des patients ou des participants. Indiquer qu'une étude relève d'un risque minimal n'informe en rien sur le risque maximal auquel on peut exposer les patients/participants. Il précise uniquement les procédures administratives à devoir suivre selon qu'on relève ou non d'un des cas de figures.

Il s'ensuit qu'il est difficile de savoir s'il faut ou non un Comité de Protection des Personnes. Si on reprend Dupont \citeyearpar{Dupont2019}, on pourrait considérer qu'un arrêté se situe tout en bas de la hiérarchie des normes, et qu'il n'est respectable que si lui-même respecte les normes qui lui sont supérieures. Cela signifie que, si l'arrêté est en contraduction, ou au moins incohérent avec la loi, il serait légitime de ne pas le respecter. Cependant, cette attitude ne correspond sans doute ni à l'esprit de la loi, ni à l'application du principe de précaution que tout chercheur devrait faire prévaloir avant de mener une recherche.

Face à cet imbroglio, la Fédération des comités d'éthique de la recherche a établi une liste de critères qui permettent aux chercheurs de qualifier leur recherche comme relevant ou non de recherche impliquant la personne humaine (RIPH). On retrouve ce document \href{https://www.federation-cer.fr/media-files/28420/questionnaire-pour-auto-qualification_9-juin-2020_doc-2.docx}{ici}. Dans tous les cas, il semble opportun de demander a minima un avis à un comité d'éthique de la recherche. La plupart des universités en ont un à présent.

Faire appel à un CPP ou à un CER est un moyen d'avoir un regard extérieur sur nos pratiques afin d'attirer l'attention sur les points qui pourraient être problématiques. Cependant, contrairement à ce que Dupont \citeyearpar{Dupont2019} suggère de manière implicite en écrivant ``la nécessité d'avoir un avis favorable d'un CPP pour toute publication dans une revue scientifique référencée'', nous ne pensons pas qu'on puisse considérer qu'un avis favorable d'un CER ou d'un CPP puisse être considéré comme un laissez-passer éthique pour publier dans des revues scientifiques.
Tout projet de recherche qui est sur le point de commencer a pour vocation d'être in fine publié, même si, dans les faits, ce n'est pas toujours le cas. Ainsi, cela concerne l'ensemble des projets. Il ne faudrait pas comprendre que, ayant des données intéressantes, il soit possible de demander après coup un avis à un comité d'éthique pour pouvoir publier ces données. L'avis doit être obtenu au préalable, car si un manquement éthique est survenu, il n'est plus possible de le prévenir. Par ailleurs, on ne peut pas non plus imaginer qu'un chercheur ou une chercheuse puisse se dire ``je vais mener une recherche A que je voudrais publier et une recherche B que je n'ai pas vocation à publier, je vais donc demander un avis uniquement pour le projet A''. Cela laisserait toute lattitude pour mener des projets non éthiques sous prétexte qu'ils n'ont pas vocation à être publiés. La situation en serait même paradoxales : mener une recherche non éthique pour faire avancer la science est inacceptable, mais mener une recherche non éthique qui a en plus le désavantage de ne pas faire avancer la science serait encore plus inacceptable. Bref, on comprend que la finalité de publication ne peut et ne doit en aucune manière être un critère pour demander un avis à un comité d'éthique ou à un comité de protection des personnes.

Une objection qui pourrait nous être faite est qu'il n'est matériellement pas possible pour les CPP et les CER d'évaluer les centaines de projets que mènent les étudiants de psychologie dans le cadre de leur mémoire de recherche. C'est tout à fait vrai et il ne faudrait pas mettre ces étudiant·es en difficulté. Il faut parfois des délais importants pour obtenir une réponse d'un CPP et un·e étudiant·e qui doit réaliser son mémoire n'a pas le loisir d'attendre 6 mois pour commencer le recueil. Ces délais peuvent a fortiori survenir en cas ``d'errance éthique'' où un CER peut qualifier un projet relevant de la loi RIPH et le CPP considère que cela ne relève pas de leurs prérogatives. Il s'ensuit que les personnes confrontées à ce type de difficultés peuvent légitimement perdre confiance dans l'utilité et la pertinence de ces comités. Ainsi, des chercheurs peuvent, après avoir demandé un avis éthique, tenir des discours tels que : ils ne font que retarder mes recherches, sont tatillons sur des outils qu'ils ne connaissent pas\ldots{} Une perte de confiance dans les instances éthiques n'est absolument pas souhaitable, car cela pousse des chercheurs à omettre cette étape. Il est normalement possible d'éviter ce type de difficultés en suivant l'outil de qualification d'une recherche que nous avons évoqué précédemment.

Une autre objection au fait de déposer tous les projets de recherche aux comités d'éthique est qu'un·e enseignant·e-chercheur-se qui encadre de nombreux-ses étudiant·es n'a matériellement pas le temps de déposer un dossier pour chacun des projets.

Les deux remarques sont parfaitement entendables, mais plusieurs éléments viennent modérer cette vision :

\begin{enumerate}
\def\labelenumi{\arabic{enumi})}
\item
  il est possible de proposer à plusieurs étudiants la même thématique, ce qui réduit considérablement le nombre de projets à déposer ;
\item
  un projet peut être mené sur plusieurs années ;
\item
  la recherche est collaborative, ce qui implique qu'un doctorant ou un collègue peut avoir demandé un avis éthique pour le projet, ce qui a pour conséquence que la charge de ces avis éthiques ne repose pas que sur une personne ;
\item
  il est possible de proposer des projets où les données ont déjà été recueillies et qui ne nécessitent plus un nouvel avis éthique ;
\item
  il est possible de proposer des projets de revues de littérature ou de méta-analyses.
\end{enumerate}

Ainsi, même s'il serait hypocrite de considérer que demander un avis éthique pour les projets menés n'est pas chronophage, il est possible de tendre vers un avis pour l'ensemble des projets en faisant moins, mais en faisant mieux.

Par ailleurs, les membres de ces comités ne sont pas des spécialistes de vos questions de recherche et peuvent ne pas identifier les failles théoriques de votre dossier qui les amèneraient à donner un avis défavorable. Pour le formuler autrement, obtenir un avis favorable d'un comité d'éthique ou d'un CPP est une condition nécessaire à une pratique éthique de la recherche mais pas suffisante. En effet, pour illustrer notre propos, nous pouvons donner comme exemple le cas du packing. Pour l'article publié dans PlosOne en 2018, Delion \citeyearpar{Delion2018} a obtenu un avis favorable de deux instances éthiques en 2008. Si à cette époque, l'article de Spinney \citeyearpar{Spinney2007} soulignait déjà les problèmes éthiques associés à cette pratique, il semble que cet article n'était sans doute pas connu des instances qui ont examiné le dossier, sans quoi il est difficilement compréhensible qu'un avis favorable ait pu être obtenu. Plus étonnant encore était le fait que, une fois lancé, l'essai n'ait pas été stoppé malgré une déclaration consensuelle d'un panel international de spécialistes de l'autisme \citep{Amaral2011} qui écrivent au dernier paragraphe de leur article ``Nous avons atteint le consensus que les professionnels et les familles autour du monde devraient considérer cette pratique comme non éthique. De plus, cette `thérapie' ignore les connaissances acutelles relatives au trouble du spectre de l'autisme ; va à l'encontre des paramètres s'appuyant sur les preuves et les recommandations de traitement publiés aux Etats-Unis, au Canada, au Royaume-Uni, en Espagne, en Italie, en Hongrie, en Australie ; et de notre point de vue, pose un risque en empêchant ces enfants et adolescents d'accéder à leurs droits basiques en tant qu'humains que la santé et l'éducation.'' Il est important de comprendre ici également qu'il y a eu un tapage médiatique autour du packing où les associations des personnes souffrant du TSA et de leurs familles ont manifesté à l'encontre de cette pratique. Il n'était dès lors plus possible d'ignorer que ces pratiques allaient de toute évidence à l'encontre des principes éthiques de la recherche, et de la pratique clinique.

Le second cadre règlementaire auquel il faut porter attention est le Règlement Général sur la Protection des Données (RGPD). Dès lors qu'une recherche est lancée, une déclaration RGPD est nécessaire, même si dans votre jeu de données, vous considérez que les informations sont non identifiantes. La raison est simple : les RGPD ne concerne pas uniquement les données numériques. Or, pour mener une étude, un consentement libre et éclairé de la personne doit être obtenu. Son nom et sa signature sont sur ce document. Ces données sont de facto identifiantes.

Par ailleurs, toute la question est de savoir dans vos données dans quelle mesure les données sont identifiantes ou non. Poussée à l'extrême, la CNIL considère que les données seront identifiantes si une personne que vous connaissez sait que vous avez participé à une recherche, pirate les serveurs sur lesquels les données de cette recherche sont stockées et est en mesure au regard de vos caractéristiques de vous identifier pour accéder à vos résultats. Ainsi, avoir dans le jeu de données le sexe, l'âge et la profession peut suffire pour identifier une personne. Ces remarques sont importantes notamment lorsque se pose le partage des données de la recherche où les données partagées doivent être nécessaires pour traiter la question de recherche, tout en évitant de partager les données qui ne sont pas indispensables.

Il ne faut cependant pas considérer que les RGPD vous empêchent de recueillir les données dont vous avez besoin pour mener vos recherches. Elles sont là pour vous pousser à vous questionner sur la nécessité ou non d'avoir ou non de recueillir telle ou telle information et de vous questionner sur la manière dont vous allez protéger ces données. Les universités sont dotées d'un·e délégué·e à la protection des données (DPO) qui est là pour vous aider et vous guider pour vous mettre en confirmité avec la règlementation.

De prime abord, on pourrait considérer que cela représente des contraintes supplémentaires qui viennent perturber la capacité à pouvoir mener des recherches. Cependant, une manière de voir les choses est de considérer qu'un projet bien pensé en amont a aussi plus de chances d'être publié et d'avoir un impact scientifique. Ainsi, c'est une occasion pour penser votre projet, tout comme le pré-enregistrement y contribue également.

\chapter{L'équipoise clinique}\label{luxe9quipoise-clinique}

Pour atteindre cet objectif, il est nécessaire de remplir la condition d'équipoise clinique. L'équipoise clinique est atteinte lorsque, pour une population de patients au sein de laquelle les participants seront sélectionnés, les preuves cliniques à disposition ne favorisent pas l'usage d'un traitement par rapport aux autres et qu'il n'existe pas de traitement disponible en dehors de l'essai qui soit meilleur que ceux utilisés dans l'essai \citep{Freedman1987}. Respecter cette notion d'équipoise clinique permet de ne pas compromettre les intérêts des patients, et l'aléatorisation est un processus tout aussi valable que n'importe quel autre pour choisir le traitement à prodiguer.

On identifie que plusieurs points méritent notre attention.

Un premier risque auquel on s'expose est qu'adopter l'équipoise clinique en tant qu'exigeance éthique pour la recherche clinique pourrait relever de la préférence personnelle. Il est dès lors nécessaire de se fonder sur le consensus au sein de la communauté d'experts pour déterminer les traitements qui sont recommandés. Si nous en revenons à l'exemple de Lund, les experts de son époque n'étaient pas d'accord sur la manière de traiter le scorbut. Ainsi, il était acceptable de répartir les marins entre les différents traitements possibles. De cette manière, l'existence d'incertitudes au sein de la communauté d'experts semble offrir un moyen de concilier les méthodes de la recherche clinique avec les normes de la médecine clinique.

Cependant, cela ne résout pas encore tout à fait les difficultés que posent la notion d'équipoise clinique. En effet, on peut considérer que deux traitements peuvent avoir les mêmes chances de succès tout en étant plus adaptés à certains patients qu'à d'autres \citep{Gifford2007}. Par exemple, dans le cadre des psychothérapies, on peut imaginer qu'une personne appréciant l'introspection est plus susceptible de répondre favorablement à une psychanalyse qu'une personne à la recherche d'outils pratiques pour faire face à sa souffrance psychologique. Pour étendre ce raisonnement de manière plus générale, on peut considérer que certains patients peuvent avoir des préférences personnelles qui les rendent plus enclins à un traitement plutôt qu'un autre, comme préférer une procédure plus risquée en une seule fois à des procédures multiples qui sont individuellement moins risquées, mais posent le même risque collectif.

Dans le même ordre d'idées, certains traitements peuvent augmenter le risque de préjudices pour un patient donné, bien que ce traitement ait, dans sa globalité les mêmes chances de succès qu'un autre. Par exemple, chez une personne souffrant d'anxiété, lui proposer des anxiolytiques alors qu'il a un passé d'addiction repésente sans doute une moins bonne alternative qu'une TCC. Ainsi, dans un essai contrôlé randomisé, il serait potentiellement dans une condition à risque accru de préjudice. On peut néanmoins objecter qu'on peut exclure de l'essai les personnes pour lesquelles on pense qu'un des traitements peut avoir des effets secondaires négatifs. Cependant, il n'est pas toujours possible d'anticiper tous les cas de figure permettant de s'assurer qu'un patient ne sera pas dans une condition de l'étude qui lui sera défavorable.

Admettons néanmoins qu'il existe des situations où toutes les options thérapeutiques se valent, ce qui permettrait d'atteindre le critère d'équipoise clinique, un certain nombre de problèmes n'ont pas encore été résolus. En effet, Wendler \citeyearpar{sep-clinical-research} souligne encore deux problèmes que l'équipoise clinique ne permet pas d'appréhender correctement.

Premièrement, toutes les idées qui sont proposées pour envisager de nouvelles approches thérapeutiques ne sont pas bonnes et les études qui testent la preuve du concept (les premières phases des essais) ou la meilleure manière de mettre en place le dispositif thérapeutique (la phase II) vont à l'encontre des intérêts médicaux des patients puisqu'on va leur donner un traitement dont on pense qu'il sera efficace mais pour lequel il n'y a aucune certitude qu'il soit efficace, et s'il l'est, on ne connait pas encore la procédure optimale pour le mettre en oeuvre. Comme, il y a malheureusement plus de mauvaises idées que de bonnes, il est difficile de considérer que les patients prenant part aux premières phases de l'étude puissent jouir d'un quelconque bénéfice thérapeutique par rapport à un traitement efficace existant.
Avoir épuisé toutes les options thérapeutiques connues avant de proposer une nouvelle approche ne résoudrait pas le problème car, certaines approches connues peuvent être très efficaces bien qu'associées à des effets indésirables conséquents ou à des difficultés logistiques, comme un prix exorbitant, une expertise rare qui est requise pour les mettre en oeuvre, ou une durée particulièrement longue de suivi\ldots{} Il est donc parfois difficile d'épuiser ces différentes approches.

Le second point qui limite l'intérêt de l'idée d'équipoise clinique tient dans la définition de cette idée : on ne fait des essais avec des patients que s'ils sont en faveur d'un intérêt clinique , autrement dit on ne fait des études qu'avec des patients qui peuvent en tirer un bénéfice clinique \citep{sep-clinical-research}. Cependant, de nombreuses études sont réalisées auprès de populations saines, que ce soit la mise en oeuvre des premières phases des essais cliniques ou des essais pour des vaccins par exemple.Il s'ensuit que :

\begin{enumerate}
\def\labelenumi{\arabic{enumi})}
\item
  soit on étend la notion d'équipoise à quiconque prend part à une étude, indépendamment du fait qu'il soit ou non un patient. Cependant, la conclusion qui s'impose dans ce cas est que les participants sains ne devraient jamais participer à une étude puisque, étant sains, ils ne peuvent retirer un quelconque intérêt clinique et qu'on ne devrait être autorisé à faire des essais qu'avec des patients puisque seuls les malades ont un intérêt médical à participer à une étude ;
\item
  soit qu'ils sont laissés sans protection par la notion d'équipoise clinique.
\end{enumerate}

Il en résulte que l'appel à l'équipoise clinique seule ne peut pas rendre la recherche clinique compatible avec les normes de la pratique clinique.

Ainsi, un très grand nombre, si pas la majorité, des études, ne devraient pas être menées puisqu'elles posent des risques. Pour Miller et Weijer \citeyearpar{Miller2006}, il est néanmoins possible de les justifier si on distingue les recherches à visée ``thérapeutique'' des recherches à visée ``non thérapeutique''. L'idée sous-jacente à cette distinction est qu'il est légitime de se départir des normes cliniques, c'est-à-dire de la notion d'équipoise, et donc de réaliser des études qui posent des risques potentiels pour les participant·es ou pour les patient·es dans les études ``non thérapeutiques'' tant que ces études ne compromettent pas significativement leurs intérêts médicaux. Ainsi, si une étude est conceptualisée avec pour objectif d'être bénéfique aux participant·es/patient·es, ou si l'intention des chercheur·euses est d'apporter un bénéfice aux participant·es, on doit considérer l'étude comme étant thérapeutique ; À l'inverse, si l'objectif de l'étude, dans sa conception ou dans les intentions des chercheur·euses, est de recueillir des connaissances généralisables, on doit la considérer comme non-thérapeutique.

Si, en apparence, cette distinction est séduisante, nous pouvons constater toute la complexité de classifier une étude comme appartenant à une catégorie plutôt qu'à une autre, tout comme il peut être difficile de déterminer si une étude relève ou non de la loi RIPH dont nous avons parlé dans le chapitre 1. Pour illustrer notre propos, imaginons une procédure qui a pour effet de réduire les effets de l'agisme (e.g., les stéréotypes négatifs liés à l'âge qui ont un impact négatif sur un grand nombre de sphères de la vie des personnes âgées, y compris leur santé), l'idée est de pouvoir généraliser, ce n'est pas thérapeutique puisqu'il n'y a pas de malades mais cela a pour objectif d'améliorer la vie des personnes âgées, avec comme conséquence potentielle une diminution d'un certain nombre de maladies. Dans quelle catégorie, faut-il placer cette étude ? Cette difficulté est soulignée par Levine \citep{Levine1988}, qui met en avant le fait que les recherches ont, pour la grande majorité, pour objectif d'être généralisables. Sur cette base, on pourrait dès lors conclure que toutes ces recherches sont non-thérapeutiques. A l'inverse, la plupart des études ont pour objectif d'améliorer la vie des gens d'une manière ou d'une autre. Dans ce cas, on devrait considérer toutes ces études comme thérapeutiques. Le simple fait d'identifier des risques potentiels de maladie, comme un test permettant de détecter une dépression permettrait de le faire, amènerait les chercheurs à orienter les patients vers un professionnel de la santé compétent, ce qui implique dès lors que la recherche est thérapeutique. On se rend compte dès lors que cette distinction est utopique dans le cadre de la recherche clinique. Par ailleurs, Levine souligne que ce n'est pas clair de savoir pourquoi les responsabilités des chercheurs envers les patients devraient varier sur cette base. Pourquoi certains chercheurs auraient le droit d'exposer des patients à des risques au bénéfice des autres alors que d'autres n'ont pas ce droit ? Pourquoi est-ce légitime que cela soit le cas lorsque l'étude n'est pas pensée pour apporter un bénéfice direct aux participants ? Cette question est d'autant plus importante pour les publics fragiles, comme les enfants. En d'autres termes, il n'y a aucune raison de considérer que les risques auxquels nous sommes disposés d'exposer les patients ou les participants puissent varier en fonction des objectifs de l'étude. Il s'ensuit que cette distinction théorique perd de son sens dans la pratique.

Ces problèmes surviennent dès lors qu'on tente d'appliquer à la recherche clinique les normes adaptées à la médecine clinique \citep{Miller2007}. Une alternative à ce point de vue pourrait être que les obligations des soignants s'appliquent dans le contexte clinique, mais pas dans celui de la recherche. Cette distinction amène donc à rejeter la distinction thérapeutique/non thérapeutique, mais a pour inconvénient de lever les obligations que le chercheur·euse ont envers les participant·es ou les patient·es. Si les obligations du cliniciens ne s'appliquent pas à la recherche, quelles sont les obligations auxquelles sont soumis les chercheurs ? La réponse à cette question tient dans la nécessité d'établir une éthique de la recherche clinique, qui reconnait la nature inappropriée d'appliquer aux chercheurs les mêmes obligations que les soignants \citep{Miller2007}. Le point central de ce changement de référentiel est qu'il doit garantir que les participants à la recherche ne soient pas exposés à des risques excessifs sans présumer que les revendications de la médecine clinique s'appliquent aux chercheurs cliniques \citep{Emanuel2000, Eckstein2003-ii}. Le rejet de la distinction entre recherche thérapeutique et non-thérapeutique a ainsi conduit à une amélioration à la fois de la clarté conceptuelle, et des préoccupations concernant les abus potentiels envers les participants à la recherche.

\chapter{Après tout, nous sommes libres de consentir ou non.}\label{apruxe8s-tout-nous-sommes-libres-de-consentir-ou-non.}

En raison de leur formation initiale, les soignants sont souvent réticents à exposer leurs patients à des procédures risquées. Ainsi, c'est une chose d'insister sur la distinction entre recherche clinique et pratique clinique d'un point de vue philosophique, cela en est une autre d'être capable, quand on est soignant, de considérer un patient prenant part à une recherche comme étant quoi que ce soit d'autre qu'un patient \citep{sep-clinical-research}. Il n'est d'ailleurs pas rare que tant les patients que les chercheurs se positionnent par rapport à une recherche comme n'impliquant rien d'autre qu'une prestation de soins cliniques.

Si les soignants ne peuvent s'empêcher de voir les participant·es comme des patient·es, un moyen de lever cette ambiguïté serait de déplacer la responsabilité des risques encourus sur les participant·es à l'étude plutôt que sur les chercheurs. Après tout, ne pourrait-on pas considérer que les participant·es sont des êtres autonomes. Ainsi, les chercheurs devraient être autorisés à conduire des recherches et à exposer les participants à des risques, à condition qu'ils aient obtenu leur consentement et leur participation ``libre, volontaire et honnête'' \citep[page 11]{mill1989j}. Cette vision libertaire consiste à admettre que des individus compétents et avisés doivent être libres de faire ce qu'ils préfèrent, de faire ce qu'ils veulent, à condition que ceux avec lesquels ils interagissent soient compétents, informés et d'accord \citep{sep-clinical-research}. Ici, la notion de consentement libre et éclairé émerge clairement. À ce titre, de nombreuses règlementations partout dans le monde insistent sur les prérequis indispensables pour qu'un consentement soit considéré comme libre et éclairé.

Néanmoins, on ne peut décemment pas accepter qu'un chercheur puisse mener à peu près n'importe quelle recherche dès lors qu'un consentement libre et éclairé a été obtenu. A ce titre, dès la première réglementation sur la recherche clinique, à savoir la déclaration de Helsinki \citep{WMA1964}\footnote{bien que la déclaration de Helsinki s'adresse aux médecins comme l'indique le point 2 de la déclaration, elle fait écho à la législation française sur les recherches impliquant la personne humaine où des non-médecins peuvent enrichir les connaissances ``médicales'' au sens large, ce qui justifie de reproduire cette déclaration dans son intégralité dans l'Annexe. Elle fait également écho à des points de recommandations du groupe CONSORT qui fournit les éléments méthodologiques pour mener correctement un essai contrôlé randomisé.}, il est stipulé que des chercheurs ne sont autorisés à mener des recherches sur des êtres humains qu'à partir du moment où ils ont reçu l'aval d'un groupe indépendant de personnes qui s'assure que l'étude est éthiquement acceptable \citep{sep-clinical-research}.

Si les comités d'éthique ont le devoir d'identifier la valeur sociale d'une étude et les risques auxquels sont soumis les participant·es à cette étude, il s'ensuit qu'obtenir un consentement libre et éclairé n'est pas une condition suffisante pour mener l'étude. Est-ce que ces règlementations empiètent sur la capacité qu'ont les humains à faire progresser les savoirs et sur la liberté qu'a chaque individu à prendre ou non part à une étude ? En effet, comme nous l'indiquons dans l'annexe 2, au moins certaines des études qui ont essuyé un refus auraient pu avoir une valeur sociale importante. Examinons dès lors les raisons pour lesquelles ces procédures sont indispensables.

Un premier problème auquel on est confronté lorsqu'on souhaite adopter un positionnement libertaire est qu'il serait inacceptable d'inclure dans une étude une personne inadaptée à donner un consentement libre et éclairé. Ainsi, on ne devrait par exemple pas mener d'études auprès d'enfants. Le corollaire immédiat de cette logique est que les médicaments donnés aux enfants auraient uniquement été testés chez l'adulte sans s'assurer de leur efficacité et de leur sécurité auprès d'enfants \citep{Roberts2003, Field2004-an, Caldwell2004, Smith2008}, et les pédiatres seraient contraints de continuer à prodiger des soins inappropriés qu'on aurait pu éviter si les études adaptées avaient été réalisées. Ainsi, en ignorant des informations essentielles, on peut potentiellement mettre en danger la vie des enfants qu'on voulait initialement protéger.

Pour Wendler \citeyearpar{sep-clinical-research}, on peut contourner cette difficulté en considérant que cette règle vaut pour toute personne en mesure de donner son consentement éclairé, sans pour autant considérer que les recherches où les participant·es seraient inaptes à consentir seraient irrecevables. Néanmoins, dans ce dernier cas, il devient indispensable d'expliquer en quoi il est acceptable de mener ce type de recherche. En l'absence de balises concernant les études légitimes de celles qui ne le sont pas, et surtout en l'absence d'informations sur la capacité qu'ont les individus à comprendre les tenants et les aboutissants ainsi qu'à se sentir libres de consentir à participer librement à l'étude, cette position reste difficilement tenable.

À cet égard, il est utile de comprendre que, même chez des adultes considérés comme étant en mesure de prendre une décision de manière éclairée, les conditions requises pour un consentement valide seront rarement recontrées, car la grande majorité des personnes éprouvent des difficultés pour comprendre les tenants et les aboutissants d'une recherche de façon suffisante pour admettre que la décision a été prise de manière éclairée \citep{Flory2004}. Par exemple, lorsqu'on prend part à un essai contrôlé randomisé, les participants doivent comprendre que le traitement qu'ils vont recevoir dépend d'un processus qui ne prend pas en compte le traitement qui sera le plus adapté pour eux \citep{Kupst2003}, ce que la très grande majorité des participants ne comprennent pas \citep{SNOWDON1997, Appelbaum2004}.

\section{Une définition du consentement libre et éclairé.}\label{une-duxe9finition-du-consentement-libre-et-uxe9clairuxe9.}

Le consentement éclairé s'applique aux personnes en capacité de fournir leur consentement, qui ont été informées de manière honnête et transparente et qui sont volontaires pour prendre part à l'étude.

Présenté ainsi, cela semble simple, limpide et facile à mettre en oeuvre. Cependant, le point critique est que toutes les informations doivent avoir été fournies aux patient·es et que ces derniers ont parfaitement compris ces informations avant de fournir le consentement. Pour les personnes qui sont en incapacité de fournir une décision éclairée, on peut déléguer cette décision à un de leurs proches, ayant l'autorité légale de le faire.

Une absence de consentement éclairé est une négligence qui peut servir de preuve pour établir une faute professionnelle. Que faut-il dès lors pour admettre que les conditions du consentement éclairé sont pleinement remplies. Il faut que ce consentement puisse apporter :

\begin{itemize}
\tightlist
\item
  protection
\item
  autonomie
\item
  prévenir les comportements abusifs
\item
  confiance
\item
  autodétermination
\item
  une absence de soumission à l'autorité
\item
  et une intégrité personnelle
\end{itemize}

\textbf{protection}

Il faut protéger les participant·es face au zèle de certain·es chercheur/chercheuses de vouloir promouvoir la science (ou leur carrière). Il s'agit aussi de protéger les patient·es de cliniciens négligents ou trop confiants dans leur traitement. En d'autres termes, l'individu est typiquement le meilleur juge pour décider ce qui est bien ou bon pour lui \citep{Mill2003} car les intérêts d'un individu vont au-delà de l'intérêt médical. Nos croyances, par exemple, peuvent influencer nos décisions. Par exemple, un Témoin de Jéhovah qui refuse une transfusion ne prend pas une décision dans l'intérêt de sa santé. Il s'agit de respecter les règles que les personnes s'imposent. Dans ce cadre, un consentement éclairé est total lorsqu'on peut faire l'hypothèse que la décision n'est pas empreinte d'une quelconque interférence visant à contrôler la décision d'un individu, mais aussi lorsqu'il n'y a pas une mécompréhension des risques de l'intervention \citep{Beauchamp2009}.

\textbf{Autonomie}

On pourrait assez simplement justifier que le consentement est légitime lorsque l'intervention coïncide avec nos valeurs. Cette vision est néanmoins un peu simpliste : on peut prendre des décisions pleinement éclairées qui sont mauvaises pour nous (ou nos proches). Nous l'avons évoqué avec le packing. De plus, la pression sociale peut inciter les individus à faire certains choix plutôt que d'autres \citep{Dworkin1988}. De même, si nos décisions ont des conséquences négatives embarrassantes, on pourrait nier avoir été correctement informé ou nier le fait de ne pas avoir eu de pression.
Il en découle qu'il est particulièrement mal aisé de s'assurer que la décision a été prise de manière autonome.

Il ne semble dès lors pas aisé d'identifier dans quelle mesure il y a pu y avoir une quelconque interférence dans la conception qu'une personne a de son bonheur.

\section{Limites du consentement libre et éclairé.}\label{limites-du-consentement-libre-et-uxe9clairuxe9.}

Nous nous retrouvons à présent dans une situation paradoxale où on ne peut imaginer une recherche où un consentement libre et éclairé ne serait pas obtenu alors que le consentement obtenu, n'est probablement ni aussi libre, ni aussi éclairé qu'on pourrait l'espérer. Suffit-il d'informer les participants sur les tenants et aboutissants de l'étude, sur la procédure qui va être adaptée pour admettre qu'ils ont tout compris. Pour vous en faire une idée, demandez-vous s'il vous suffit d'avoir parcouru ce document une seule fois pour en avoir intégré tous les éléments ? Sans doute que non, il ne faut dès lors pas s'étonner que pour pouvoir considérer raisonnablement une recherche comme étant éthique, il faille a minima protéger les participants de ce qui pourrait leur nuire. C'est ce qu'on appelle du paternalisme.

\chapter{Le paternalisme est-il légitime ?}\label{le-paternalisme-est-il-luxe9gitime}

Si un certain paternalisme semble indispensable, il implique également d'interférer avec la liberté des agents pour leur propre bénéfice \citep{Feinberg1986}. Ce paternalisme est considéré comme étant léger si l'intervention consiste à protéger les individus de ce qui pourrait leur être nuisible. Par exemple, si on pense qu'une méthode est inefficace, voire dangereuse, il n'est pas légitime de demander à des participants de prendre part à l'étude dont la justification serait basée sur un appel à l'ignorance. Par exemple, on ne sait pas quels sont les avantages et les inconvénients d'une intoxication aux métaux lourds par rapport à une intoxication à un élément radioactif (on peut juste imaginer que les deux sont dangereux et entraînent la mort). Continuons cet exemple, et admettons que l'étude soit menée, on pourrait présenter cette étude de la manière suivante à des participant·es : ``cette étude vise à comparer les effets de métaux lourds par rapport à un élément radioactif''. Il est raisonnable de concevoir que tout le monde ne connait pas les effets des empoisonnements aux métaux lourds, ni les effets relatifs aux éléments radioactifs. Vous en doutez ? Pourtant, l'interdiction d'utiliser du plomb dans les canalisations n'est pas si ancienne, à savoir 1995. Si c'était si évident que c'est nuisible, pourquoi a-t-on ne serait-ce qu'envisagé d'utiliser ce matériel pour les canalisations ? De même, avez-vous une idée précise des éléments qui sont considérés comme des métaux lourds ? Probablement que non.

A l'inverse, la tomographie par émission de position (PETScan) s'appuie sur des traceurs radioactifs, faut-il donc interdire ces outils ? Dans cette dernière phrase, vous venez peut-être de découvrir que le PETScan s'appuyait sur des traceurs radioactifs, et votre réaction pourrait être de vous demander si c'est dangereux, d'être au moins suspicieux sur l'absence de risque de cet outil. A l'inverse, peut-être le saviez-vous, mais dans votre entourage, demandez-vous qui le sait ? Cet exemple n'a pas pour vocation de vous inciter à douter de la pertinence de cette méthode, mais de vous amener à prendre conscience que tout le monde n'a pas le même niveau de connaissances sur ce qui peut ou non être dangereux, et que la plupart des personnes laissent le soin aux experts de les conseiller sur ce qui est le mieux pour elles.

Le paternalisme peut également être beaucoup plus orienté. Il s'agit du paternalisme dur. Dans ce dernier cas, l'idée est de promouvoir les intérêts de la personne, bien que la décision qui a été prise résulte d'un choix volontaire et éclairé d'un individu compétent.

Pour illustrer d'une autre manière ces deux formes de paternalisme, on peut les transposer à l'éducation des enfants, face à un enfant qui escalade sans protection/sans sécurité un arbre, en tant que parent, on peut dire à cet enfant ``tu ne devrais pas monter sur cet arbre, car tu risques de tomber et de te blesser'' (paternalisme soft) ou ``descends tout de suite de cet arbre, tu sais que c'est interdit, je t'ai déjà dit que c'était dangereux'' (parternalisme plus dur).

Si les règlementations relatives à la recherche clinique prenaient racine dans un paternalisme dur, elles représenteraient une inacceptable restriction à l'action autonome \citep{sep-clinical-research}. En revanche, on ne peut considérer un paternalisme plus doux comme une restriction libertaire à partir du moment où les données indiquent que nombre de participant·es échouent à saisir les tenants et les aboutissants de la recherche à laquelle on leur demande de participer.

Par ailleurs, même si nous omettions le fait que beaucoup de participants n'aient pas suffisamment compris l'étude pour donner un consentement valide, il n'en découlerait pas qu'il soit légitime de ne pas réguler la recherche clinique, car ce qu'on peut infliger à autrui n'est pas régi par ce que cette autre personne consente à ce que nous lui infligions. Certaines personnes peuvent accepter d'être torturées, d'être battues, est-ce que cela justifie qu'on le fasse ? A fortiori dans un contexte de la recherche ? Probablement que vous répondrez par la négative, du moins je l'espère, mais peut-être aussi vous dites-vous que des chercheurs ne feraient pas cela sous prétexte que la personne a donné son accord. Pour essayer de vous détromper, deux exemples me paraissent éclairants.

Le premier exemple date de 1974. Marina Abramović est une artiste qui va réaliser une expérience artistique intitulée Rythm 0. La performance artistique consiste à proposer au public 72 objets, parmi lesquels on retrouve du miel, un fouet, un scalpel ou une arme à feu et le public est libre de les utiliser comme il le souhaite sur le corps de l'artiste. Au début de la performance, les interactions sont plutôt timides, mais vont progressivement devenir de plus en plus violentes et agressives. A la fin des 6h qu'a duré sa performance, l'artiste était nue, recouverte de griffures et de coupures. Cet exemple montre que, si des participant·es sont disposé·es à ce qu'on leur fasse subir ce genre de choses, il serait présomptueux de prendre pour acquis qu'aucun chercheur et aucune chercheuse ne pourrait les leur infliger.

Le second ensemble d'exemples pour vous détromper porte sur des problèmes éthiques posés par des recherches qui ont été réalisées. Il en existe une multitude. A titre d'exemple, à l'heure actuelle, des milliers de participants à des recherches voient leurs droits humains violés en Amérique Latine \citep{Homedes2014}. Plus proche de chez nous, en France, Didier Raoult a été épinglé pour un certain nombre de problèmes éthiques, notamment par Elisabeth Bik. Celle-ci dénonce, sur les aspects éthiques uniquement, ce qu'on pourrait appeler de la science néocolonialiste \citep{Bik1}, c'est-à-dire des recherches menées par des chercheurs issus de pays riches dans des pays moins favorisés pour collecter des données et publier les résultats en l'absence (ou avec peu d'implications) des chercheurs locaux. Elle dénonce par ailleurs le fait qu'un certain nombre de ses recherches n'avaient pas les autorisations éthiques appropriées \citep{Bik2}. Si vous êtes un adepte d'une approche libertaire, peut-être considéreriez-vous que les manquements à l'éthique sont relativement mineurs jusqu'à présent.
Le problème, en l'occurrence, est qu'un certain nombre d'études impliquaient la participation de populations particulièrement vulnérables, à savoir des enfants et des sans-abris. Ainsi, dans le cas présent, ce qu'Elisabeth Bik explique de manière limpide, c'est que les populations vulnérables doivent être protégées contre toute forme de coercition et d'influence indue. La coercition signifie qu'une personne pourrait avoir le sentiment qu'elle ne pourrait pas accéder à des choses de base, telles que l'accès à des services de soin, à moins de participer à l'étude. Pour comprendre pleinement la problématique dans le cas de Didier Raoult, il faut prendre conscience que, si on aborde les sans-abris à l'entrée des centres d'hébergement, ils pourraient craindre que, s'ils ne prennent pas part à l'étude, on ne leur permette pas d'accéder au centre d'hébergement. Transposez à une situation plus générale, imaginez que vous deviez vous faire soigner d'une maladie grave, trouveriez-vous normal qu'on vous dise ``nous sommes disposés à vous soigner qu'à la condition que vous preniez part à notre étude'' ? Probablement que non. Dans cette situation, accepteriez-vous bon gré mal gré d'y prendre part ? Probablement que oui, car vous voudrez être soigné. On imagine ici aisément qu'un sans-abris ait le souhait d'avoir un lit pour la nuit. La participation à l'étude n'est donc plus une participation sur la base d'un consentement libre et éclairé, mais sur la base d'une forme de coercition.

On comprend ici mieux pourquoi des précautions supplémentaires sont nécessaires pour mener des recherches sur ces populations particulièrement vulnérables. Quand on constate que, malgré les règles en vigueur, des chercheurs les contournent, demandez-vous ce qui se passerait si elles n'existaient pas ! En résumé, on peut affirmer qu'il est nécessaire de traiter autrui de manière appropriée, indépendamment du fait de savoir s'ils/elles consentent ou non à ce qu'on les traite autrement.

Certains pourraient encore objecter que, même si certaines recherches sont éthiquement problématiques, les régulations qui existent font peu de sens pour la plupart des études qui impliquent des personnes en mesure de donner un consentement. En fait, il faut comprendre que ces règles ne sont pas seulement là pour protéger de manière paternaliste les participant·es contre les dommages que pourraient infliger la participation à la recherche, même si ce point est évidemment essentiel, elles sont surtout là pour fixer des limites : les limites relatives aux dommages que des chercheur·euses peuvent infliger en traitant de manière inappropriée leurs participant·es, mais aussi des limites quant à l'approbation et aux bénéfices que peut tirer la société de ce type d'études \citep{sep-clinical-research}. Pour le formuler autrement, les régulations ne sont pas seulement pour éviter que des participant·es soient exposé·es à des risques de blessures sans qu'il n'y ait de bonnes raisons, il faut également que les chercheur·euses ne puissent pas les exposer à ces risques sans qu'il y ait des raisons convaincantes, et la société ne devrait pas apporter son soutien et ne devrait pas tirer de bénéfices à le faire.

Cette vision montre les connexions fortes qu'on peut entrevoir avec la manière dont les obligations des cliniciens restreignent le type de recherche qu'ils/elles peuvent conduire. Si la dyade clinicien/patient n'est pas la dyade chercheur/participant, si nous concluons que les normes de la recherche clinique doivent être différentes des normes qui prévalent pour les clinicien·nes, alors la question qui reste est d'être en mesure de fixer les limites de ce que peuvent ou non infliger des chercheur·euses aux participants à des recherches.

Wendler \citeyearpar{sep-clinical-research} fournit un exemple particulièrement éclairant pour comprendre tous les enjeux qui sous-tendent ces questionnements, que nous nous permettons de reproduire. On sait que les maltraitances physiques et émotionnelles causent énormément de souffrances aux personnes qui en sont victimes. Imaginons qu'une équipe de recherche mette en place un design expérimental dont l'objectif est de pouvoir prendre en charge les personnes qui sont victimes de maltraitances. Dans cette recherche, l'idée est de reproduire avec des participant·es des conditions de maltraitance durant une semaine et ces participants seront répartis de manière aléatoire dans différentes conditions expérimentales qui correspondent à différentes méthodes de prise en charge des victimes de maltraitance.

Bien que la question posée soit indubitablement importante et que les participant·es ne prennent part à l'étude qu'après avoir donné un consentement éclairé, cela ne signifie pas que l'étude est éthiquement acceptable. D'abord, il faudrait se demander si les conditions expérimentales imitent suffisamment les conditions de vie réelles pour que l'étude ait une bonne validité externe. Il faudrait aussi se demander s'il n'existe pas des manières moins risquées de mener cette étude. Enfin, même si la réponse à ces deux question justifierait la réalisation de l'étude, il resterait à déterminer s'il est éthiquement acceptable que des chercheur·euses puissent traiter leurs participant·es de la sorte. Un peu près toute personne se questionnant sur l'éthique devrait considérer que ce n'est pas le cas. Cette conclusion suggère que l'éthique de la recherche, à la fois sur la manière dont elle est pratiquée et sur la manière dont elle devrait être pratiquée, va au-delà du respect de l'autonomie qu'a un individu car elle inclut les standards quant aux comportements que les chercheurs et les chercheuses ont le droit et le devoir d'adopter. Ce point est non seulement la pierre angulaire de l'éthique de la recherche, mais également le défi le plus important qu'elle a à relever.

\chapter{Participants et risques auxquels ils ont exposés}\label{participants-et-risques-auxquels-ils-ont-exposuxe9s}

Nous avons évoqué à plusieurs reprises le fait d'identifier les risques auxquels nous sommes disposés d'exposer les personnes qui prennent part à une recherche. À peu près tout le monde est d'accord sur le fait qu'il ne serait pas acceptable d'exposer les participant·es à des risques sans qu'un bénéfice en résulte. Pour le formuler autrement, l'idée est qu'il est acceptable d'exposer autrui à des risques à la condition que cela bénéficie au plus grand nombre. En dehors des conditions de recherche, on peut même aller jusqu'à considérer que nous sommes disposés à exposer les autres à des risques si nous pouvons en tirer un bénéfice. Par exemple, si vous vous rendez en voiture au supermarché, vous exposez vos voisins à un risque de pollution pour que vous puissiez faire vos courses. De même, les ambulances exposent les piétons à un plus grand risque d'être fauchés au profit de la personne qu'elles transportent. Les entreprises du tabac exposent les fumeurs à un risque accru de cancer au bénéfice de leur compte en banque.

Bref, nous constatons que nous sommes constamment soumis à des risques de différentes natures que nous acceptons sans y prêter la moindre attention. Dès lors, quand on parle de risques dans le cadre de la recherche, de quoi parle-t-on ?

Bien qu'on puisse considérer qu'il y ait de grandes similarités entre les risques auxquels nous sommes exposés dans notre quotidien, une recherche serait éthiquement problématique si on exposait les participant·es à ce type de risques \citep{Wilson2010}. A l'inverse, cela ne viendrait à l'esprit de personne de demander à une oeuvre caricative d'avoir un accord d'une entité indépendante, d'obtenir le consentement éclairé des bénéficiaires de l'oeuvre sur la base d'une description exhaustive des risques et potentiels bénéfices, son propos, sur la durée, ses objectifs et les moyens pour y arriver. En cela, on peut parler d'une certaine forme d'exceptionnalisme. Ainsi, pour un certain nombre d'auteurs, les recommandations éthiques dans le domaine de la recherche clinique ne sont en aucune manière justifiées \citep{Sachs2009, Wertheimer2010}. Certains auteurs avancent même que ces régulations sont la plus grande menace aux avancées scientifiques, en particulier dans le domaine médical \citep{Stewart2008} où bureaucratie qui résulte de l'ensemble des régulations pourrait menacer la créativité que la science requiert \citep{Sullivan2008}. Il me semble que, pour bien comprendre ces auteurs, le reproche qui est fait n'est pas de vouloir s'assurer que les recherches respectent les principes éthiques, mais questionnent la charge administrative que l'ensemble des mesures de protection engendrent. Ainsi, si on s'inscrit dans cette perspective, l'idée est de considérer l'éthique de la recherche clinique comme n'importe quelle autre activité de la vie quotidienne où on expose des personnes à des risques au bénéfice d'autres personnes.

Par exemple, les ouvriers qui manipulent de l'amiante sont exposés à des risques. Cela ne va pas à l'encontre de l'éthique dès lors qu'ils sont d'accord pour faire ce travail et qu'ils reçoivent un salaire juste en contrepartie. On pourrait donc imaginer que les personnes qui prennent part à une étude pourraient également recevoir une contrepartie financière pour leurs efforts. Cette vision des choses est beaucoup moins libertaire que celle décrite plus haut puisqu'elle implique de traiter les participants avec équité, et de ne pas les exploiter. Pour aller plus loin, rémunérer les participant·es, c'est bien, les rémunérer beaucoup, c'est mieux.

Cependant, rémunérer les participants pourrait exacerber plus que résoudre les problèmes éthiques, notamment parce que cela pourrait représenter une incitation indue à participer et cela objectifierait les participants \citep{sep-clinical-research}. En effet, on comprend aisément ici que les personnes les plus précaires pourraient devenir des participant·es ``professionnels'', s'exposant à des risques importants pour bénéficier du défraiement qui résulte de la participation à l'étude.

Ainsi, si les partisans de l'exceptionnalisme ont pointé à juste titre la question de savoir en quoi la recherche clinique exposerait indûment les participant·es à des risques qui n'existeraient pas dans les autres activités de la vie, la comparaison avec les autres activités de la vie ne permet pas d'en déduire des recommandations concrètes. Pour continuer le rapprochement avec les risques auxquels sont exposés les travailleurs d'une entreprise, le Code du travail en France n'est pas le même que celui de l'Allemagne, du Royaume-Uni ou des États-Unis. Par exemple, certaines législations imposent un salaire minimum et d'autres non ; certains travailleur·ses sont protégés contre les discriminations ou les licenciements abusifs, d'autres non\ldots{} On ne peut imaginer que l'éthique de la recherche clinique soit dépendante de différences de politiques. On ne peut non plus imaginer que les personnes qui prennent part à des études puissent être considérées comme des objets qu'il serait légitime de manipuler à sa guise dès lors qu'on a obtenu leur consentement et qu'on leur octroie un défraiement.

Une autre vision des choses est de renverser la perspective et de considérer que nous sommes toutes et tous, d'une manière ou d'une autre, bénéficiaires des avancées scientifiques des générations précédentes : que ce soit par les vaccins, par les traitements médicamenteux, par les outils de correction perceptifs\ldots{} Il s'ensuit que nous sommes redevables à ces générations précédentes en agissant de la même manière pour les générations futures : nous sommes contractuellement engagés à prendre des risques au bénéfice des autres \citep{Heyd1996, Harris2005}.

Cette idée est plutôt bien retransmise dans le film ``un monde meilleur'' de Mimi Leder où, pour rendre le monde meilleur, un enfant propose d'aider 7 personnes qui en ont vraiment besoin. Cette aide ne demande rien en retour, si ce n'est que les personnes qui en bénéficient doivent à leur tour aider 7 personnes qui en ont vraiment besoin. Dans l'absolu, on pourrait considérer que cet arrangement est tout à fait acceptable et profitable au plus grand nombre.

Cependant, si nous sommes obligés à l'encontre d'autrui, ce n'est pas à l'encontre des générations futures, mais des générations précédentes et participer à une recherche ne nous libère pas de nos obligations à l'égard des personnes qui ont permis d'améliorer notre futur.

Si participer à des recherches ne nous libère pas de la reconnaissance que nous pouvons avoir pour les générations précédentes, on peut la voir comme une manière d'avoir de la reconnaissance envers le système social dans son ensemble, système auquel appartient la recherche clinique \citep{Brock1992}. Ainsi, si on grandit dans une société qui s'appuie sur un schéma de coopération et d'entraide, nous devons faire notre part, puisque nous avons pu bénéficier de nombreux avantages à vivre dans un tel environnement. On peut reprocher à ce raisonnement que le simple fait de jouir d'un certain nombre de bénéfices sans avoir donné notre accord pour y contribuer dans le futur ne permet pas de comprendre clairement en quoi cela obligerait l'individu à aider les autres. Pour le formuler autrement, ce n'est pas parce que je vous ai rendu service hier sans que vous ne me l'ayez demandé que vous vous retrouvez dans une situation de me rendre la pareille aujourd'hui parce que je vous en fais la demande. Pour comprendre toute la portée de ce raisonnement, on peut prendre le cas des enfants. Quiconque a déjà vu des (petits) enfants bénéficier d'un examen médical ou d'un vaccin sait à quel point ils s'y opposent vigoureusement. Comme ils refusent de bénéficier des avantages de la société, en quoi devraient-ils lui être redevables ? Bref, le raisonnement contractuel ne tient plus dès lors que la personne qui bénéficie du système social refuse, quel qu'en soit le moyen, d'accepter ces bénéfices.

Bien que la parallèle ne soit pas parfait (car il pose un grand nombre d'autres questions qui sont au-delà du raisonnement que nous essayons d'expliquer), le cas des impôts peut aider à comprendre ce raisonnement : il existe une sorte de contrat entre l'Etat et les citoyens selon lequel l'accès à toute une série de services est disponible (presque) gratuitement, car leur financement est assuré par les impôts payés par la communauté. Pourtant, peu de personnes seraient disposées à payer plus d'impôts (pour pouvoir bénéficier de plus de services), et il est même raisonnable de penser que la plupart des personnes serait favorable à payer moins d'impôts et de taxes, sans qu'on diminue les services auxquels elles ont accès. Ici arrive le point critique : pourquoi devrait-on payer pour des services dont on ne bénéficie pas ? Par exemple, les personnes qui n'ont pas d'enfants à charge (et n'en auront jamais) pourraient se dire que, si leurs impôts ne servaient pas à payer les frais de scolarisation des enfants, ils paieraient moins tout en pouvant continuer à bénéficier, par exemple du remboursement des frais de santé. Á l'inverse, les personnes qui sont rarement malades pourraient adopter le raisonnement inverse. Ici, dans cet exemple, nous pouvons identifier qu'il y a conflit entre le contrat qui est imposé (payer des impôts) et les bénéfices que les usagers en tirent, ce qui amène certaines personnes à dénoncer ce contrat ou à ne pas le respecter (au travers de la fraude fiscale par exemple). Á partir du moment où on n'accepte pas le contrat, quelle qu'en soit la manière et les raisons, les théories fondées sur ce principe sont mises en difficulté \citep{Gauthier1990-GAUMDC-3}.

On pourrait toujours avancer que participer à des études relève d'une obligation à partir du moment où nous ne savons quelle position nous allons occuper dans la société dans le futur \citep{Rawls99}. Si je reprends l'exemple des impôts évoqué précédemment. La personne qui refuse de payer pour le remboursement des frais de santé, car il/elle ne tombe jamais malade, peut dans le futur avoir un grave accident qui va nécessiter une longue hospitalisation et des frais médicaux extrêmement importants. Cette vision est appelée le voile d'ignorance, qui peut se résumer par ``on ne sait de quoi l'avenir sera fait''. La justification principale sous-tendant cet argument est que, dans une société, il est indispensable de trouver un arrangement équitable \citep{Rawls99}. Ainsi, si la société fournit une structure équitable entre les bénéfices et le fardeau qui incombe à chacun, les membres de cette société ne peuvent plus avancer que la distribution coût/bénéfice est inéquitable. Cependant, il est possible d'abandonner cette société pour une autre. Le droit de partir suggère que l'équité du système n'entraîne pas \emph{de facto} une obligation de participer à des études \citep{sep-clinical-research}. Cela autorise, et ce n'est pas rien, ceux et celles qui souhaitent prendre part à des études à le faire sans qu'on puisse dire que c'est injuste d'exposer ces personnes à des essais cliniques. On pourrait même considérer que, non seulement ce n'est pas injuste, mais c'est même raisonnable pour n'importe quelle personne de prendre part à une étude, y compris les personnes qui ne sont pas en mesure de fournir un consentement éclairé \citep{sep-clinical-research}.

Puisqu'on ne peut pas imposer de prendre part à la recherche, la question qui se pose est de savoir ce qui pousse une personne à prendre part à un essai clinique. Il est raisonnable de penser que ce sont des considérations morales qui nous y poussent. Si c'est le cas, on peut se demander s'il est éthique d'exposer des personnes qui ne sont pas en mesure de fournir un consentement éclairé, puisqu'il est également raisonnable de penser dans ce cas que ces personnes ne sont pas en mesure d'appréhender les motivations morales qui pourraient justifier ou non de prendre part à une étude. Si ce n'est pas le cas, on ne peut pas faire appel au voile d'ignorance pour justifier les études avec des personnes qui ne peuvent consentir librement \citep{sep-clinical-research}. En revanche, si ce n'est pas éthique, alors le voile d'ignorance peut difficilement justifier ces études, et par extension la recherche clinique de manière plus générale.

Une alternative à ce dilemme moral, qui s'appuie aussi sur l'idée du voile d'ignorance, est que les personnes qui prennent part à une étude le font uniquement dans leur propre intérêt, et non pour des questions morales. Le problème est qu'on peut être confronté à des problèmes quand il s'agit de se demander si une étude est éthique ou non. Les recherches dont le bénéfice global est le plus élevé pourraient être celles qui sont en même temps les moins éthiques. Par exemple, pendant la crise COVID-19, trouver rapidement un remède ou un vaccin avait le potentiel de sauver des millions de vie. Cependant, peu de personnes auraient trouvé acceptable de tester un vaccin en exposant volontairement des participants au virus SARS-COV-19. Pourtant, les personnes qui s'appuient sur le voile d'ignorance sur la base d'un intérêt personnel pourraient trouver cette étude acceptable en s'appuyant sur l'idée que le coût/bénéfice global permettra de sauver des millions de vies. Le risque auquel on expose une minorité est clairement surpassé par le potentiel de sauver des millions de vies.
Selon cette approche, on peut donc considérer que, comme malheureusement des millions de personnes étaient susceptibles de décéder de la covid, exposer volontairement certains individus pour tester l'efficacité d'un traitement ou d'un vaccin pourrait se justifier, car si on ne fait rien, si on ne trouve pas de traitement, alors beaucoup d'autres personnes continueront à être infectées et à décéder. Il faut donc choisir la stratégie qui permet de réduire au maximum les infections et les risques de décès. Ce qu'il adviendra des personnes qui ont été volontairement infectées n'est manifestement pas une question à laquelle une approche basée sur le voile d'ignorance semble apporter une réponse. On comprend dès lors qu'on peut difficilement se satisfaire d'un tel raisonnement.

Une autre difficulté de cette approche est plus spécifique au domaine de la psychologie. On se rend compte qu'il est possible de saisir assez aisément ce raisonnement fondé sur le voile d'ignorance lorsqu'on pense à des essais cliniques dans le domaine médical. En revanche, qu'en est-il en psychologie. La plupart des troubles psychiques ne sont pas mortels en soi. De même, on ne peut infecter une personne avec une dépression ou un trouble anxieux. Bien sûr, les participants pourraient toujours y trouver un bénéfice s'ils sont atteints d'un trouble, à savoir celui de trouver un moyen de se sentir mieux. Néanmoins, la notion de coût/bénéfice devient moins claire. Compter un nombre de personnes décédées fournit une réponse simple à la question de savoir si le bénéfice social est grand. En revanche, quand le trouble n'est pas mortel, quel est le coût pour la société ? Le nombre de jours d'absence au travail ? Le bien-être global ? La consommation des substances psychoactives ? Et quand sera-t-il acceptable de proposer une opération du cerveau, des traitements à base d'électrochocs ou de bains glacés (qui sont des traitements qui ont vraiment existé et dont certains continuent à être utilisés) ? Quelle niveau de torture admet-on pour 100 jours d'absentéisme en moins ? Qu'en est-il pour les participants sains. Est-il légitime de leur faire une IRM ? Pourrait-on faire une résection du cerveau pour tester la plasticité cérébrale que pourrait générer un nouveau traitement qui permettrait de contrer les effets des maladies neurodégénératives ?

Nous comprenons ici que nous sommes disposés à exposer autrui à des risques mais ces risques, doivent être minimaux. Il y a un accord implicite que les risques ne doivent pas être trop élevés. Certains auteurs avancent même qu'il ne devrait n'y avoir aucun risque du tout, y compris pour les personnes adultes en mesure de fournir un consentement éclairé. S'il y a consensus sur le fait de limiter au maximum les risques auxquels on expose les participants, il n'y a pas de consensus pour savoir quels sont les risques acceptables ou non. Quand peut-on considérer qu'un risque est suffisamment faible ?

Le défi que doivent relever les chercheurs est d'identifier les standards fiables pour que, dans un contexte donné, le risque soit suffisamment faible. On comprend ici qu'une question annexe émerge : est-ce légitime de faire varier les risques acceptables en fonction du contexte et/ou de la population qui sera étudiée ?

Essayons d'abord de cerner ce que représente un risque suffisamment faible. Pour Nicholson \citeyearpar{Nicholson1986-NICMRW}, un risque faible lorsque le risque de subir une blessure sérieuse est négligeable. Selon cette vision des choses, le risque auquel on expose des enfants quand on utilise des questionnaires n'est pas plus inquiétant que de légèrement l'agacer pendant quelques minutes. Dans ce cas, exposer des participants à ce type de risque ne semble pas être à l'origine d'un quelconque problème éthique, dès lors que cela procurerait un bénéfice à d'autres.

Les études pour lesquelles les risques sont réellement faibles pourraient néanmoins être plus rares qu'en apparence. Dans le domaine médical, par exemple, les pratiques routinières comme une prise de sang ne peuvent rentrer dans les critères de risques très faibles. En psychologie, proposer un nouveau type de prise en charge d'un trouble peut potentiellement occasionner une détérioration de l'état des participants. Passer des tests évaluant l'humeur ou le fonctionnement cognitif d'une personne peut être le déclencheur d'une prise de conscience d'un trouble. Par exemple, une personne pourrait ne pas se questionner sur ses accomplissements et être amenée à ruminer face à une question du type ``choisissez ce qui vous correspond le mieux :

\begin{enumerate}
\def\labelenumi{\arabic{enumi})}
\tightlist
\item
  Je n'ai pas le sentiment d'avoir échoué
\item
  J'ai le sentiment que j'ai échoué plus que la plupart des gens
\item
  Quand je regarde ce que j'ai accompli dans ma vie, je ne vois qu'une accumulation d'échecs
\item
  J'ai le sentiment que ma vie complète est un échec''
  {[}item librement traduit de question de dépression de Beck \citeyearpar{Beck2011-ph}{]}
\end{enumerate}

Une autre manière de définir les risques comme étant suffisamment faibles est de les trouver acceptables tant qu'ils ne dépassent pas les risques auxquels les individus sont exposés lors des examens routiniers \citep[\citet{Resnik2005}]{Kopelman2000}. Cette recommandation fournit un seuil clairement identifiable pour fixer ce qui est ou non un risque acceptable. Le problème avec ce raisonnement est que les risques auxquels sont exposés les individus sains dans les pratiques médicales routinières sont si faibles que cela exclut de réaliser un certain nombre de recherches qu'on pourrait considérer intuitivement comme acceptables. Par ailleurs, comme les techniques médicales deviennent plus sûres et moins invasives, il s'ensuit que des recherches qui étaient acceptables à un moment deviennent inacceptables par la suite \citep{sep-clinical-research}.

Puisqu'on ne peut pas s'appuyer sur les risques auxquels on est exposé dans le contexte médical, on peut envisager les choses sous un autre angle : il est commun de considérer que la recherche clinique est éthiquement acceptable tant que les risques nets ne dépassent pas ceux auxquels un individu est exposé dans sa vie quotidienne. Ainsi, il serait légitime de mener des études, y compris avec des enfants, dès lors que les risques sont négligeables étant donné qu'il y a très peu de chance de les exposer à un dommage quelconque, mais cela implique néanmoins de garantir une compensation dès lors qu'une blessure sérieuse survient \citep{VanEys1978-VANROC-10}.

Il est tout à fait vrai que si nous n'étions pas disposés à accepter certains risques, nous ne pourrions pas réaliser nos activités de la vie quotidienne. Pensez à une situation aussi banale que de traverser la route : on peut être renversé par une voiture, éclabousser par un véhicule qui roule dnas une flaque d'eau, glisser sur une plaque de verglas, trébucher dans une crevasse, croiser une voiture de police/un camion de pompiers/une ambulance qui nous assourdit avec leur sirène, on peut être bousculer par un autre piéton, on peut être victime d'un objet tombant du toit d'une des habitations, on peut avoir un oiseau qui fait ses besoins sur nous\ldots{} et cette liste est encore longue. Pourtant, tous ces risques sont relativement faibles et, excepté lorsque certains stimuli particuliers attirent notre attention, nous les mettons en arrière-plan pour ne pas surcharger notre esprit avec des risques auxquels nous pourrions être exposés, mais qui ont une très faible probabilité de survenir.

Le bât blesse dans ce raisonnement lorsqu'on le réexamine de plus près : nous sommes disposés à prendre des risques pour pouvoir vaquer à nos occupations de la vie quotidienne. Ainsi, quand on participe à une étude, on s'expose à des risques supplémentaires à ceux auxquels on serait normalement exposé. Par ailleurs, nous tirons bénéfice des activités que nous menons, c'est pour cela que nous acceptons ces risques. Nous sommes donc dans une impasse éthique si nous exposons des personnes à ces mêmes risques sans qu'elles puissent en tirer bénéfice, puisque le bénéfice que nous tirons en prenant des risques dans notre vie quotidienne est compensé par les bénéfices que nous tirons à pouvoir réaliser nos activités quotidiennes \citep{Friedman2006}.

Par ailleurs, la manière dont nous ignorons les risques auxquels nous nous exposons dans notre vie quotidienne n'est pas un processus purement rationnel. Par exemple, lorsqu'une personne est dans l'incapacité de le faire ignorer, cela va entraîner une grande source de détresse psychologique, qui est communément décrite sous l'appellation de phobie. Les personnes agoraphobiques ainsi que les personnes phobiques sociales sont ainsi des exemples de personnes qui n'arrivent pas à évaluer que la probabilité de survenue des risques qui les terrifient est (très) faible. Á l'inverse, certaines personnes sont enclines à faire de nouvelles expériences et à prendre beaucoup de risques. Ainsi, la perception qu'a une personne d'une situation est différente en fonction de ses caractéristiques, de la perception de contrôle qu'on a sur la situation, ainsi que de la familiarité avec la situation \citep[voir par exemple,][]{Weinstein1989}. Au-delà des différences interindividuelles, même si on s'attarde aux risques qui seraient évalués de manière purement rationnelle et objective, nous ne prêtons pas attention à tous les risques potentiels, car cela aurait pour conséquence que les effets seraient plus délétères que de ne pas y prêter attention, comme dans le cas des phobies que nous venons de décrire. Ce point est critique d'un point de vue éthique : on ne peut pas s'appuyer sur le fait que le coût associé au fait de porter notre attention sur un risque donné dans nos activités quotidiennes est plus grand que les bénéfices pour l'individu pour justifier les risques auxquels nous pouvons exposer des participants dans les recherches au bénéfice des autres. Pour de nombreuses activités de notre vie quotidienne, il y a des risques d'accident, n'importe quel bricoleur, cuisinier, jardinier le sait. Le risque de se blesser gravement dans les activités quotidiennes doit représenter un signal d'alarme pour considérer que ce standard puisse être utilisé comme niveau de risques acceptable d'une expérimentation. En effet, si nous réalisons un nombre suffisamment grand d'expérimentations, en s'appuyant sur les probabilités des risques auxquels nous sommes exposés dans notre vie quotidienne, cela signifierait que certains participants décéderaient et que d'autres souffriraient de handicap à vie.

Une alternative plus raisonnable serait que les risques minimums acceptables sont ceux qui n'augmentent pas les risques auxquels les participants sont exposés. L'hypothèse sous-jacente est que les risques auxquels on est exposé en prenant part à une recherche ne doivent pas être accrus par rapport à ceux auxquels il/elle aurait été exposé·e si il ou elle avait réalisé son activité quotidienne.Par exemple, pour prendre part à l'étude, la personne doit prendre sa voiture pour conduire. Si elle n'y avait pas pris part, elle aurait pris sa voiture pour un autre trajet ou aurait réalisé une autre activité risquée. Cependant, les risques auxquels on est exposé quand on prend part à une recherche sont additifs plutôt que soustractifs. Reprenons, l'exemple du trajet : si la personne doit également aller faire des courses. Sans la participation à l'expérience, elle part de chez elle, va au supermarché, revient. Avec l'expérimentation, elle part de chez elle, va au laboratoire où la recherche est menée, revient, prend sa liste de courses et va au supermarché, avant de revenir. Les risques de faire un accident de voiture sont donc doublés ici. Par ailleurs, nous acceptons d'être exposés à certains risques, car les activités associées à ces risques nous apportent des bénéfices personnels. Quels sont les bénéfices qu'on en tire s'il n'y a aucune chance d'en tirer un bénéfice médical, comme c'est le cas dans les premières phases d'une étude. Bref, il ne faudrait inclure que des participants qui seraient exposés à des risques plus importants s'ils n'avaient pas pris part à l'étude, ce qui, de manière pratique, est intenable.

Cela a amené certains auteurs à soutenir que le progrès qu'offre la recherche clinique est, d'un point de vue normatif, optionnel, alors que protéger les individus de tout risque de blessure auxquels la recherche clinique les expose est une priorité absolue \citep{Jonas1976}.

Il faut néanmoins comprendre que le point de vue de Jonas n'implique pas que la recherche clinique est nécessairement non éthique, mais que les conditions où il est légitime de mener de telles études sont très strictes. D'après lui, c'est une bonne chose de pouvoir soigner ``les petits maux'' qui surviennent quand on vit, ou quand on vieillit, comme l'arthrite, mais trouver un traitement à ces problèmes ne répond pas à des problèmes profonds dans notre vie. Ainsi, d'après Jonas, si c'est tout ce que peut offrir la recherche clinique, il est raisonnable d'être réticent à accepter la multitude de risques auxquels il faut exposer les participants pour atteindre ces objectifs.

Cet argument est acceptable tant qu'on considère que le \emph{status quo} est acceptable. En revanche, dès lors qu'on souffre d'un mal pour lequel il serait possible de trouver une solution en y consacrant les moyens nécessaires ne seraient probablement pas d'accord avec cette vision. Les jugement relatifs à la situation actuelle de la société ne prend en compte la société qu'à un niveau très général mais ignore la multitude de personnes en situation de souffrance. C'est sans contexte la souffrance de ces personnes qui justifient le plus la nécessité de mener des recherches cliniques. Il ne faut néanmoins pas mécomprendre Jonas : il ne prétend pas qu'il n'existe plus de maladie mais que la gravité ou la prévalence de ces maladies ne justifient pas de mener des recherches cliniques. La crise COVID vient contester sans appel ce point de vue. Il ne faut pas non plus négliger le fait que cesser de faire de la recherche, c'est ne pas développer des outils qui pourront être utiles à un moment où il faudra trouver des solutions nouvelles à des problèmes auxquels nous n'avons pas encore été confrontés. Combien de temps aurait-il fallu pour développer un vaccin à ARN si des études n'avaient pas déjà examiné la faisabilité de cette méthode avant que la crise COVID ne survienne ?

En revanche, on peut accorder au crédit de Jonas deux points importants. Premièrement, il n'est pas impensable qu'une étude peu éthique ait des répercussions sur la société dans son intégralité puisque les recherche sont menées pour la société et en son nom. À ce titre, la position de Dr.~Raoult durant la crise covid en est une parfaite illustration : ses études peu éthiques ont jeté le discrédit sur les résultats des études sérieuses, en particulier sur les bénéfices du vaccin. Le second point qu'il souligne est qu'exposer des participants à certains risques au bénéfice des autres pourrait nous amener sur une pente glissante où il y aurait de sérieux abus dans la société. C'est bien pour cela que les comités d'éthiques jouent un rôle central dans la légitimation des études cliniques.

Une autre manière d'interpréter Jonas est de considérer qu'il fait une distinction entre actif et passif. L'idée est qu'il y a une différence morale profonde entre le fait de causer activement du mal par rapport à simplement ne pas empêcher que cela ne survienne. En d'autres termes, c'est différent de tuer quelqu'un par rapport à ne pas empêcher qu'une personne ne meurt. Étant donné que les chercheurs peuvent exposer de manière active leurs participants à des risques de nuisance, cela signifie que, quand ces nuisances surviennent, cela implique que le chercheur a été actif dans le fait de créer cette nuisance. Puisque ces recherches cliniques sont réalisées au bénéfice de la société, cela implique que la société est complice de ces nuisances causées à autrui. Ne pas mener de recherche a en contrepartie pour conséquence de permettre à des personnes de souffrir de maladies qu'il aurait possible de guérir. Dans cette situation, la vertu de n'avoir rien fait n'implique pas d'avoir clairement mal agi.

Plusieurs problèmes surviennent avec ce raisonnement. Le premier est que la population a le droit de penser que les décideurs font tout ce qui est en leur pouvoir pour les protéger et s'il y a des manquements alors il peut être tenu pour responsable. L'affaire du Mediator en est une illustration où l'agence nationale de sécurité du médicament (donc une agence de l'état) a été condamnée pour ``blessures et homicides involontaires''.

Un autre problème avec la vision de Jonas est que la recherche scientifique permet de trouver de nouvelles alternatives de traitements qui sont plus sures pour les patients. Pour illustrer ce propos, on peut s'intéresser à l'impact des effets indésirables des médicaments : entre 1994 et 1998, les effets secondaires de traitements représentaient entre la 4ème et le 6ème cause de décès aux USA \citep{Lazarou1998}. Pourtant, une grande partie des hospitalisation (et sans doute des décès) dus à ces effets indésirables auraient pu être évités \citep{Giardina2018}. On comprend ici que la question n'est plus d'exposer des personnes à des risques pour améliorer un petit peu la vie des autres mais trouver un compromis entre les risques auxquels on expose les participants pour diminuer le risque auxquels les cliniciens exposent leurs patients en raison de l'imperfection des traitements utilisés. Ainsi, l'argument de Jonas devrait alors être compris comme le fait qu'il est moralement mal de causer des torts à des participants durant une recherche clinique mais que ce ne serait pas le cas si un clinicien cause (involontairement) des torts à des patients. On comprend ici que l'argument tient difficilement.

Néanmoins, il est important de comprendre que les critiques de Jonas ne sont pas non plus sans fondement. Le risque fondamental est que les chercheurs ne prennent aucune précaution quant aux raisons pour lesquelles les participants prennent part à l'étude alors même que leurs objectifs peuvent être très différents, voire opposés. Il ne faudrait pas dans ce cas traiter les participants comme s'ils n'avaient pas de but précis pour prendre part à l'étude, et il ne faudrait pas considérer que les raisons pour lesquelles les personnes prennent part à l'étude soient écartées trop facilement comme si elle étaient sans importance.

Ainsi, pour Jonas, une étude qui est éthiquement acceptable est une étude où les participants partagent les buts de la recherche. En considérant que que c'est dans l'intérêt d'une personne de réaliser des objectifs, il s'ensuit que, en prenant part à l'étude, participants agissent dans leur propre intérêt malgré le fait d'être exposés à des procédures potentiellement risquées.

Pour les personnes qui souffrent d'une maladie particulière, prendre part à une recherche qui permet de trouver de nouveaux remèdes pour cette maladie constitue indubitablement un objectif qu'ils peuvent avoir en commun avec le chercheur afin de potentiellement pouvoir guérir. C'est dans leur intérêt de prendre part à l'étude. De manière plus générale, ce qui relève de nos propres intérêts dépend de ce qu'on veut, de ce qu'on préfère, ou des valeurs que l'on veut promouvoir \citep{Griffin1988}. Cette vision des choses permet de justifier la recherche clinique car elle promeut les intérêts de l'individu tant que ces recherches informent correctement les participants et que ces derniers souhaitent prendre part à l'étude après avoir reçu cette information, indépendamment d'avoir ou non la maladie investiguée.

En effet, au-delà d'être atteint de la maladie qui est investiguée, il peut y avoir une multitude de raisons pour lesquelles une personne peut soutenir les objectifs d'une recherche : avoir un membr de sa famille qui a la maladie, considérer que c'est important de une meilleure connaissance du monde\ldots{} Faut-il dès lors s'assurer que les participants adhèrnent aux objectifs d'une étude pour pouvoir justifier une étude ? Les guides de bonnes pratiques ne considèrent presque jamais que cela doit être une condition nécessaire, car un consentement libre et éclairé de personnes en mesure de comprendre les objectifs de l'étude suppose que les personnes adhèrent aux objectifs de l'étude.

\chapter{Conclusions}\label{conclusions}

La question éthique à laquelle nous avons tenté de répondre visait à savoir ce qui justifiait d'exposer des participants à certains risques liés aux recherches au bénéfice des autres. Si ces recherches permettent d'améliorer la santé, d'améliorer les traitements, d'améliorer la prévention des études et que les participants partagent ces objectifs, il est raisonnable de considérer ces recherches comme acceptables

Il va de soi qu'à ce point, la question n'est pas encore épuisée, loin s'en faut, mais on peut avoir une idée des questions qu'il est bon de se poser lorsqu'on veut mener une recherche.

Parmi les points qui n'ont pas été abordés, il y a notamment les recherches cliniques menées par l'industrie pharmaceutique dont les objectifs pourraient être différents que d'améliorer la santé des gens. En effet, des enjeux financiers, souvent colossaux, viennent poser de nouvelles question éthiques et morales. Ainsi, en plus de transformer les difficultés éthiques inhérentes à la recherche clinique, les recherches soutenues financièrement par l'industrie a le pouvoir de transformer la manière dont beaucoup de problèmes éthiques sont traités dans ce contexte. Á titre d'exemple, il est raisonnable de se demander comment le fait de pouvoir gagner d'énormes quantité d'argent entre en conflit avec la mise en place des dispositifs appropriés pour protéger les participants \citep{Fontanarosa2005}.

Les régulations qui se sont mises progressivement en place avaient pour objectif de mieux protéger les participants car les pratiques dans lesquelles les cliniciens menaient en même temps leurs propres essais pouvaient mettre en danger leurs patients, patients qui, bien souvent, n'avaient pas consience qu'ils étaient impliqués dans une recherche. Ces problèmes ont été à l'origine de la séparation entre pratique clinique et recherche clinique et aux législations que nous avons abordées. Ces changements ont permit non seulement de mieux protéger les participants mais aussi de mener des recherches plus sophistiquées, mieux pensées \citep{sep-clinical-research}.

Cependant, mener correctement une étude peut prendre des années, années pendant lesquelles les cliniciens sont confrontés à des patients qui remplissent les conditions d'intérêt de la recherche. Les informations relatives à ces patients, c'est-à-dire le traitement reçu, la dose et les effets secondaires observés sont enregistrées dans le dossier médical du patient. On peut imaginer qu'en étant systématique, ces données représentent une source précieuse d'informations qui permettraient d'informer sur les pratiques futures. C'est pourquoi certaines personnes ont prôné le dévleoppement de systèmes de santé qui apprennent afin d'avoir une amélioration continue et des innovations constantes où les nouvelles connaissances seraient obtenues directement par l'expérience de soin \citep{institute2013best}.

Cette approche aurait le mérite de réconcilier pratique clinique et recherche clinique et apporterait une réponse au fait de justifier les risques auxquels on expose les participants. Cependant, cela ne lève pas le problème de la recherche avec les individus sains, et cela pose d'autres difficultés : comment s'assurer que cela ne réintroduise des risques potentiels d'exploitation des participants. Ainsi, on peut se demander si les patients impliqués dans de tels systèmes doivent être avertis que leurs données pourraient être utilisées à des fins de recherche ? Si oui, quand ? Á quels risques supplémentaires pourrait-on exposer les participants/patients par rapport aux pratiques de soin standard ?

\chapter{Références}\label{ruxe9fuxe9rences}

\chapter*{Annexes}\label{annexes}
\addcontentsline{toc}{chapter}{Annexes}

\section{Annexe 1 : Version française de la déclaration de Helsinki dans sa version de 2008.}\label{annexe-1-version-franuxe7aise-de-la-duxe9claration-de-helsinki-dans-sa-version-de-2008.}

\textbf{Préambule}

\begin{enumerate}
\def\labelenumi{\arabic{enumi}.}
\tightlist
\item
  L'Association Médicale Mondiale (AMM) a élaboré la Déclaration d'Helsinki comme un énoncé de principes éthiques applicables à la recherche médicale impliquant des êtres humains, y compris la recherche sur du matériel biologique humain et sur des données identifiables.
\end{enumerate}

La Déclaration est conçue comme un tout indissociable. Chaque paragraphe doit être appliqué en tenant compte de tous les autres paragraphes pertinents.

\begin{enumerate}
\def\labelenumi{\arabic{enumi}.}
\setcounter{enumi}{1}
\tightlist
\item
  Conformément au mandat de l'AMM, cette Déclaration s'adresse en priorité aux médecins. L'AMM invite cependant les autres personnes engagées dans la recherche médicale impliquant des êtres humains à adopter ces principes.
\end{enumerate}

\textbf{Principes généraux}

\begin{enumerate}
\def\labelenumi{\arabic{enumi}.}
\setcounter{enumi}{2}
\item
  La Déclaration de Genève de l'AMM engage les médecins en ces termes : « La santé de mon patient prévaudra sur toutes les autres considérations » et le Code International d'Ethique Médicale déclare qu'un «médecin doit agir dans le meilleur intérêt du patient lorsqu'il le soigne».
\item
  Le devoir du médecin est de promouvoir et de sauvegarder la santé, le bien-être et les droits des patients, y compris ceux des personnes impliquées dans la recherche médicale. Le médecin consacre son savoir et sa conscience à l'accomplissement de ce devoir.
\item
  Le progrès médical est basé sur la recherche qui, en fin de compte, doit impliquer des êtres humains.
\item
  L'objectif premier de la recherche médicale impliquant des êtres humains est de comprendre les causes, le développement et les effets des maladies et d'améliorer les interventions préventives, diagnostiques et thérapeutiques (méthodes, procédures et traitements). Même les meilleures interventions éprouvées doivent être évaluées en permanence par des recherches portant sur leur sécurité, leur efficacité, leur pertinence, leur accessibilité et leur qualité.
\item
  La recherche médicale est soumise à des normes éthiques qui promeuvent et assurent le respect de tous les êtres humains et qui protègent leur santé et leurs droits.
\item
  Si l'objectif premier de la recherche médicale est de générer de nouvelles connaissances, cet objectif ne doit jamais prévaloir sur les droits et les intérêts des personnes impliquées dans la recherche.
\item
  Il est du devoir des médecins engagés dans la recherche médicale de protéger la vie, la santé, la dignité, l'intégrité, le droit à l'autodétermination, la vie privée et la confidentialité des informations des personnes impliquées dans la recherche. La responsabilité de protéger les personnes impliquées dans la recherche doit toujours incomber à un médecin ou à un autre professionnel de santé, et jamais aux personnes impliquées dans la recherche, même si celles-ci ont donné leur consentement.
\item
  Dans la recherche médicale impliquant des êtres humains, les médecins doivent tenir compte des normes et standards éthiques, légaux et réglementaires applicables dans leur propre pays ainsi que des normes et standards internationaux. Les protections garanties par la présente Déclaration aux personnes impliquées dans la recherche ne peuvent être restreintes ou exclues par aucune disposition éthique, légale ou réglementaire, nationale ou internationale.
\item
  La recherche médicale devrait être conduite de sorte qu'elle réduise au minimum les nuisances éventuelles à l'environnement.
\item
  La recherche médicale impliquant des êtres humains doit être conduite uniquement par des personnes ayant acquis une éducation, une formation et des qualifications appropriées en éthique et en science. La recherche impliquant des patients ou des volontaires en bonne santé nécessite la supervision d'un médecin ou d'un autre professionnel de santé qualifié et compétent.
\item
  Des possibilités appropriées de participer à la recherche médicale devraient être offertes aux groupes qui y sont sous-représentés.
\item
  Les médecins qui associent la recherche médicale à des soins médicaux devraient impliquer leurs patients dans une recherche uniquement dans la mesure où elle se justifie par sa valeur potentielle en matière de prévention, de diagnostic ou de traitement et si les médecins ont de bonnes raisons de penser que la participation à la recherche ne portera pas atteinte à la santé des patients concernés.
\item
  Une compensation et un traitement adéquats doivent être garantis pour les personnes qui auraient subi un préjudice en raison de leur participation à une recherche.
\end{enumerate}

\textbf{Risques, contraintes et avantages}

\begin{enumerate}
\def\labelenumi{\arabic{enumi}.}
\setcounter{enumi}{15}
\tightlist
\item
  Dans la pratique médicale et la recherche médicale, la plupart des interventions comprennent des risques et des inconvénients.
\end{enumerate}

Une recherche médicale impliquant des êtres humains ne peut être conduite que si l'importance de l'objectif dépasse les risques et inconvénients pour les personnes impliquées.

\begin{enumerate}
\def\labelenumi{\arabic{enumi}.}
\setcounter{enumi}{16}
\tightlist
\item
  Toute recherche médicale impliquant des êtres humains doit préalablement faire l'objet d'une évaluation soigneuse des risques et des inconvénients prévisibles pour les personnes et les groupes impliqués, par rapport aux bénéfices prévisibles pour eux et les autres personnes ou groupes affectés par la pathologie étudiée.
\end{enumerate}

Toutes les mesures destinées à réduire les risques doivent être mises en œuvre. Les risques doivent être constamment surveillés, évalués et documentés par le chercheur.

\begin{enumerate}
\def\labelenumi{\arabic{enumi}.}
\setcounter{enumi}{17}
\tightlist
\item
  Les médecins ne peuvent pas s'engager dans une recherche impliquant des êtres humains sans avoir la certitude que les risques ont été correctement évalués et pourront être gérés de manière satisfaisante.
\end{enumerate}

Lorsque les risques s'avèrent dépasser les bénéfices potentiels ou dès l'instant où des conclusions définitives ont été démontrées, les médecins doivent évaluer s'ils continuent, modifient ou cessent immédiatement une recherche.

\textbf{Populations et personnes vulnérables}

\begin{enumerate}
\def\labelenumi{\arabic{enumi}.}
\setcounter{enumi}{18}
\tightlist
\item
  Certains groupes ou personnes faisant l'objet de recherches sont particulièrement vulnérables et peuvent avoir une plus forte probabilité d'être abusés ou de subir un préjudice additionnel.
\end{enumerate}

Tous les groupes et personnes vulnérables devraient bénéficier d'une protection adaptée.

\begin{enumerate}
\def\labelenumi{\arabic{enumi}.}
\setcounter{enumi}{19}
\tightlist
\item
  La recherche médicale impliquant un groupe vulnérable se justifie uniquement si elle répond aux besoins ou aux priorités sanitaires de ce groupe et qu'elle ne peut être effectuée sur un groupe non vulnérable. En outre, ce groupe devrait bénéficier des connaissances, des pratiques ou des interventions qui en résultent.
\end{enumerate}

\textbf{Exigences scientifiques et protocoles de recherche}

\begin{enumerate}
\def\labelenumi{\arabic{enumi}.}
\setcounter{enumi}{20}
\item
  La recherche médicale impliquant des êtres humains doit se conformer aux principes scientifiques généralement acceptés, se baser sur une connaissance approfondie de la littérature scientifique, sur d'autres sources pertinentes d'informations et sur des expériences appropriées en laboratoire et, le cas échéant, sur les animaux. Le bien-être des animaux utilisés dans la recherche doit être respecté.
\item
  La conception et la conduite de toutes les recherches impliquant des êtres humains doivent être clairement décrites et justifiées dans un protocole de recherche.
\end{enumerate}

Ce protocole devrait contenir une déclaration sur les enjeux éthiques en question et indiquer comment les principes de la présente déclaration ont été pris en considération. Le protocole devrait inclure des informations concernant le financement, les promoteurs, les affiliations institutionnelles, les conflits d'intérêts potentiels, les incitations pour les personnes impliquées dans la recherche et des informations concernant les mesures prévues pour soigner et/ou dédommager celles ayant subi un préjudice en raison de leur participation à la recherche.

Dans les essais cliniques, le protocole doit également mentionner les dispositions appropriées prévues pour l'accès à l'intervention testée après l'essai clinique.

\textbf{Comités d'éthique de la recherche}

\begin{enumerate}
\def\labelenumi{\arabic{enumi}.}
\setcounter{enumi}{22}
\tightlist
\item
  Le protocole de recherche doit être soumis au comité d'éthique de la recherche concernée pour évaluation, commentaires, conseils et approbation avant que la recherche ne commence. Ce comité doit être transparent dans son fonctionnement, doit être indépendant du chercheur, du promoteur et de toute autre influence indue et doit être dûment qualifié. Il doit prendre en considération les lois et réglementations du ou des pays où se déroule la recherche, ainsi que les normes et standards internationaux, mais ceux-ci ne doivent pas permettre de restreindre ou exclure l'une des protections garanties par la présente Déclaration aux personnes impliquées dans la recherche.
\end{enumerate}

Le Comité doit avoir un droit de suivi sur les recherches en cours. Le chercheur doit fournir au Comité des informations sur le suivi, notamment concernant tout évènement indésirable grave. Aucune modification ne peut être apportée au protocole sans évaluation et approbation par le Comité. A la fin de la recherche, les chercheurs doivent soumettre au Comité un rapport final contenant un résumé des découvertes et des conclusions de celle-ci.

\textbf{Vie privée et confidentialité}

\begin{enumerate}
\def\labelenumi{\arabic{enumi}.}
\setcounter{enumi}{23}
\tightlist
\item
  Toutes les précautions doivent être prises pour protéger la vie privée et la confidentialité des informations personnelles concernant les personnes impliquées dans la recherche.
\end{enumerate}

\textbf{Consentement éclairé}

\begin{enumerate}
\def\labelenumi{\arabic{enumi}.}
\setcounter{enumi}{24}
\item
  La participation de personnes capables de donner un consentement éclairé à une recherche médicale doit être un acte volontaire. Bien qu'il puisse être opportun de consulter les membres de la famille ou les responsables de la communauté, aucune personne capable de donner un consentement éclairé ne peut être impliquée dans une recherche sans avoir donné son consentement libre et éclairé.
\item
  Dans la recherche médicale impliquant des personnes capables de donner un consentement éclairé, toute personne pouvant potentiellement être impliquée doit être correctement informée des objectifs, des méthodes, des sources de financement, de tout éventuel conflit d'intérêts, des affiliations institutionnelles du chercheur, des bénéfices escomptés et des risques potentiels de la recherche, des désagréments qu'elle peut engendrer, des mesures qui seront prises après l'essai clinique et de tout autre aspect pertinent de la recherche. La personne pouvant potentiellement être impliquée dans la recherche doit être informée de son droit de refuser d'y participer ou de s'en retirer à tout moment sans mesure de rétorsion. Une attention particulière devrait être accordée aux besoins d'informations spécifiques de chaque personne pouvant potentiellement être impliquée dans la recherche ainsi qu'aux méthodes adoptées pour fournir les informations. Lorsque le médecin ou une autre personne qualifiée en la matière a la certitude que la personne concernée a compris les informations, il doit alors solliciter son consentement libre et éclairé, de préférence par écrit. Si le consentement ne peut pas être donné par écrit, le consentement non écrit doit être formellement documenté en présence d'un témoin.
\end{enumerate}

Toutes les personnes impliquées dans des recherches médicales devraient avoir le choix d'être informées des conclusions générales et des résultats de celles-ci.

\begin{enumerate}
\def\labelenumi{\arabic{enumi}.}
\setcounter{enumi}{26}
\item
  Lorsqu'il sollicite le consentement éclairé d'une personne pour sa participation à une recherche, le médecin doit être particulièrement attentif lorsque cette dernière est dans une relation de dépendance avec lui ou pourrait donner son consentement sous la contrainte. Dans ce cas, le consentement éclairé doit être sollicité par une personne qualifiée en la matière et complètement indépendante de cette relation.
\item
  Lorsque la recherche implique une personne incapable de donner un consentement éclairé, le médecin doit solliciter le consentement éclairé de son représentant légal. Les personnes incapables ne doivent pas être incluses dans une recherche qui n'a aucune chance de leur être bénéfique, sauf si celle-ci vise à améliorer la santé du groupe qu'elles représentent, qu'elle ne peut pas être réalisée avec des personnes capables de donner un consentement éclairé et qu'elle ne comporte que des risques et des inconvénients minimes.
\item
  Lorsqu'une personne considérée comme incapable de donner un consentement éclairé est en mesure de donner son assentiment concernant sa participation à la recherche, le médecin doit solliciter cet assentiment en complément du consentement de son représentant légal. Le refus de la personne pouvant potentiellement être impliquée dans la recherche devrait être respecté.
\item
  La recherche impliquant des personnes physiquement ou mentalement incapables de donner leur consentement, par exemple des patients inconscients, peut être menée uniquement si l'état physique ou mental empêchant de donner un consentement éclairé est une caractéristique nécessaire du groupe sur lequel porte cette recherche.
\end{enumerate}

Dans de telles circonstances, le médecin doit solliciter le consentement éclairé du représentant légal. En l'absence d'un représentant légal et si la recherche ne peut pas être retardée, celle-ci peut être lancée sans le consentement éclairé. Dans ce cas, le protocole de recherche doit mentionner les raisons spécifiques d'impliquer des personnes dont l'état les rend incapables de donner leur consentement éclairé et la recherche doit être approuvée par le comité d'éthique de la recherche concerné. Le consentement pour maintenir la personne concernée dans la recherche doit, dès que possible, être obtenu de la personne elle-même ou de son représentant légal.

\begin{enumerate}
\def\labelenumi{\arabic{enumi}.}
\setcounter{enumi}{30}
\item
  Le médecin doit fournir des informations complètes au patient sur la nature des soins liés à la recherche. Le refus d'un patient de participer à une recherche ou sa décision de s'en retirer ne doit jamais nuire à la relation patient-médecin.
\item
  Pour la recherche médicale utilisant des tissus ou des données d'origine humaine, telles que les recherches sur tissus et données contenues dans les biobanques ou des dépôts similaires, les médecins doivent solliciter le consentement éclairé pour leur analyse, stockage et/ou réutilisation. Il peut se présenter des situations exceptionnelles où il est impraticable, voire impossible d'obtenir le consentement. Dans de telles situations, la recherche peut être entreprise uniquement après évaluation et approbation du comité d'éthique de la recherche concerné.
\end{enumerate}

\textbf{Utilisation de placebo}

\begin{enumerate}
\def\labelenumi{\arabic{enumi}.}
\setcounter{enumi}{32}
\tightlist
\item
  Les bénéfices, les risques, les inconvénients, ainsi que l'efficacité d'une nouvelle intervention doivent être testés et comparés à ceux des meilleures interventions avérées, sauf dans les circonstances suivantes :
\end{enumerate}

lorsqu'il n'existe pas d'intervention avérée, l'utilisation de placebo, ou la non-intervention, est acceptable ; ou

lorsque pour des raisons de méthodologie incontournables et scientifiquement fondées l'utilisation de toute intervention moins efficace que la meilleure éprouvée, l'utilisation d'un placebo, ou la non-intervention, est nécessaire afin de déterminer l'efficacité ou la sécurité d'une intervention,

et lorsque les patients recevant une intervention moins efficace que la meilleure éprouvée, un placebo, ou une non-intervention, ne courent pas de risques supplémentaires de préjudices graves ou irréversibles du fait de n'avoir pas reçu la meilleure intervention éprouvée.

Le plus grand soin doit être apporté afin d'éviter tout abus de cette option.

\textbf{Conditions de l'accès à l'intervention testée après l'essai clinique}

\begin{enumerate}
\def\labelenumi{\arabic{enumi}.}
\setcounter{enumi}{33}
\tightlist
\item
  En prévision d'un essai clinique, les promoteurs, les chercheurs et les gouvernements des pays d'accueil devraient prévoir des dispositions pour que tous les participants qui ont encore besoin d'une intervention identifiée comme bénéfique dans l'essai puissent y accéder après celui-ci. Cette information doit également être communiquée aux participants au cours du processus de consentement éclairé.
\end{enumerate}

\textbf{Enregistrement des recherches, publication et dissémination des résultats}

\begin{enumerate}
\def\labelenumi{\arabic{enumi}.}
\setcounter{enumi}{34}
\item
  Toute recherche impliquant des êtres humains doit être enregistrée dans une banque de données accessible au public avant que ne soit recrutée la première personne impliquée dans la recherche.
\item
  Les chercheurs, auteurs, promoteurs, rédacteurs et éditeurs ont tous des obligations éthiques concernant la publication et la dissémination des résultats de la recherche. Les chercheurs ont le devoir de mettre à la disposition du public les résultats de leurs recherches impliquant des êtres humains. Toutes les parties ont la responsabilité de fournir des rapports complets et précis. Ils devraient se conformer aux directives acceptées en matière d'éthique pour la rédaction de rapports. Les résultats aussi bien négatifs et non concluants que positifs doivent être publiés ou rendus publics par un autre moyen. La publication doit mentionner les sources de financement, les affiliations institutionnelles et les conflits d'intérêts. Les rapports de recherche non conformes aux principes de la présente Déclaration ne devraient pas être acceptés pour publication.
\end{enumerate}

\textbf{Interventions non avérées dans la pratique clinique}

\begin{enumerate}
\def\labelenumi{\arabic{enumi}.}
\setcounter{enumi}{36}
\tightlist
\item
  Dans le cadre du traitement d'un patient, faute d'interventions avérées ou faute d'efficacité de ces interventions, le médecin, après avoir sollicité les conseils d'experts et avec le consentement éclairé du patient ou de son représentant légal, peut recourir à une intervention non avérée si, selon son appréciation professionnelle, elle offre une chance de sauver la vie, de rétablir la santé ou d'alléger les souffrances du patient. Cette intervention devrait par la suite faire l'objet d'une recherche pour en évaluer la sécurité et l'efficacité. Dans tous les cas, les nouvelles informations doivent être enregistrées et, le cas échéant, rendues publiques.
\end{enumerate}

\section{Annexe 2 : Rôle scientifique des comités d'éthique.}\label{annexe-2-ruxf4le-scientifique-des-comituxe9s-duxe9thique.}

Il est légitime ici de se questionner sur le rôle scientifique qu'a un Comité d'éthique. D'un côté, les membres de ces comités d'éthique ne sont pas experts du domaine des projets qu'ils évaluent. Nous l'avons précédemment dans l'encart 1 avec le cas particulier du packing dans l'autisme. Ainsi, ces comités d'éthique ne sont le plus souvent pas en mesure d'identifier les failles théoriques qui mettent en péril l'évaluation du dossier, ainsi que les erreurs méthodologiques qui mettent en péril la capacité qu'ont les chercheurs de pouvoir répondre adéquatement à la question qu'ils veulent poser, en d'autres termes, la validité de l'étude.

Par ailleurs, la question de la valeur sociale d'une étude est sans doute sujette à interprétation. Une belle illustration de ce phénomène est cette \href{https://www.sciencealert.com/these-8-papers-were-rejected-before-going-on-to-win-the-nobel-prize}{liste} de chercheurs et chercheuses qui ont essuyé des refus avant d'être acceptés dans une autre revue et d'obtenir un prix Nobel. Certes, la question n'est pas éthique ici. Cependant, si des chercheurs compétents ne sont pas en mesure d'identifier la valeur sociale d'une étude dans leur domaine, on peut aisément inférer que des personnes qui ne sont pas dans le domaine auront d'autant plus de difficultés pour répondre à cette prérogative.

En résumé, qu'une décision favorable soit accordée à une étude qui n'aurait pas dû être menée ou qu'une décision défavorable soit adressée à une étude qui a du mérite, on comprend que ces deux types d'erreurs amènent à se questionner sur la pertinence d'accorder à un comité d'éthique le soin d'évaluer la valeur sociale d'une étude.

Il ne s'agit pas ici de remettre en question le fait que la valeur sociale d'une étude est indispensable à prendre en compte pour déterminer la pertinence éthique de mener ou non une étude, a fortiori lorsque celle-ci implique des personnes. En revanche, il s'agit de se questionner sur la pertinence de la procédure qui fait actuellement référence. La conséquence directe de cette constatation est que les comités d'éthique limitent le plus souvent leur expertise à identifier une exposition potentielle à des risques et à déterminer si cela est ou non justifié. A ce titre, il est intéressant de constater que la déclaration d'Helsinki engage la responsabilité des chercheur·ses mais aussi des éditeurs scientifiques, pour s'assurer que les études ont été menées avec les plus hauts standards des règles d'éthique.

\bibliography{book.bib}

\end{document}
